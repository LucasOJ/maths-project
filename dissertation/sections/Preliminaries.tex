\section{Preliminaries}
\label{Prelims}

\subsection{Notation}

First we introduce notation for expressing graph metrics. This notation will also be used throughout the paper. % TODO - GET RID OF SECOND LINE?

\begin{definition}
	Complement Set

	\noindent
	Given a graph $G = (V, E)$, for a set of vertices $S \subseteq V$ we define the complement set $\comp{S} = V \setminus S$
\end{definition}

\begin{definition}
	Cut Set

	\noindent
	Given a graph $G = (V, E)$, for a set of vertices $S \subseteq V$ we define the cut set $ E(S, \comp{S}) = \left\{\{u, v\} \in E \mid u \in S, v \in \comp{S} \right\}.$
\end{definition}

\begin{definition}
	Degree of a vertex

	\noindent
	Given a graph $G = (V, E)$, let $d_v$ be the number of neighbouring nodes $v$ is adjacent to, i.e. $$
		d_v = |\left\{ e \in E : v \in e \right\}|
	$$
\end{definition}

\begin{definition}
	Volume of a vertex set

	\noindent
	Given a graph $G = (V, E)$ with $S \subseteq V$, let 
	$$
		\text{vol}(S) = \sum_{v \in S} d_v
	$$
\end{definition}

% TODO: Explain Implications of Cut set defn?

\subsection{Graph Metrics}

\begin{definition}
	Conductance of a graph $G = (V, E)$

	$$
		\Phi(G) = \min_{\emptyset \neq S \subset V} \frac{|E(S, \comp{S})|}{\min\{\text{vol}(S), \text{vol}(\comp{S})\}}
	$$
\end{definition}

\begin{definition}
	Diligence of a cut $ E(S, \comp{S}) $
	$$
		\rho(S) = \comp{d}(S) \min_{\{u, v\} \in E(S, \comp{S}) } \left\{ \max \left\{ \frac{1}{d_u},\frac{1}{d_v} \right\} \right\}
	$$ 

	where $\comp{d}(S) := \frac{\sum_{v \in S} d_v}{|S|}$ is the average degree of the vertices in $S$
\end{definition}

\begin{definition}
	Diligence of a graph $G$
	$$
		\rho(G) = \min_{S \in V} \rho(S) 
	$$
\end{definition}

\subsection{Introducing Dynamic Networks}

Here we formally define the Dynamic Network structure rumours will spread on.

\begin{definition}
	Dynamic Network

	\noindent
	A dynamic network is a sequence of graphs $\mathcal{G} = (G_t)_{t \in \mathbb{N}}$ indexed by an integer time $t$. All the graphs in the sequence have the same vertex set at each time step, but the edge set may vary, i.e.  $G_t = (V, E_t)$ where $E_t$ is some edge set on $V$.
\end{definition}

$G_t$ represents the topology of the network at the discrete time step $t$. However, asynchronous rumour spreading algorithms operate in continuous time, so we need to define the topology of the network at non-integer times. To represent the state of the network at any non-negative continuous time $\gamma \in \mathbb{R}_+$, we say that the current network topology $G_\gamma$ := $G_{\floor\gamma}$. Thus, for any time $\gamma \in [t, t + 1)$ the network topology is fixed to $G_t$. This corresponds to allowing the network topology change at integer time steps only.

% TODO: Segway

\begin{definition}
	Vertex degree at time $\gamma \in \mathbb{R}_+ $ 

	\noindent
	For a Dynamic Network $\mathcal{G}$ on a vertex set $V$, $d_v(\gamma)$ is the degree of a vertex $v \in V$ at time $\gamma$, i.e. the degree of $v$ in $G_\gamma$
\end{definition}


\subsection{Review of Poisson Processes}

In this subsection we introduce Poisson processes, which will be needed to specify the first asynchronous rumour spread algorithm.

The Poisson process ...

TODO

\begin{enumerate}
	\item Definition
	\item $N(s + t) - N(s)$ has a Poisson distribution with rate $\lambda t$
	\item Superposition
\end{enumerate}


\subsection{Order statistics of exponential random variables}

TODO: Prove that minimum of exponentials is exponential, where the probability each exponential that attains the minimum is dependent on each exponentials' rate, and independent of the value of the min


\subsection{Stochastic Domination}
Coupling proof of stochastic domination of poisson


% TODO: Discuss implications for async rumour spreading - interpreting poission process as exponential clock, not interested in value of the process

