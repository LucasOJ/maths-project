\section{Proof that the asynchronous rumour spreading time bound is almost tight}\label{section:asyncBoundTight}
\label{section:AsyncLowerBound}

In this section, we prove that the upper bound derived in Section \ref{AsyncUpperBoundSection} is almost tight up to scaling in $n$, by constructing an example of a dynamic network given in \cite{asyncPaper}, where the spread time is close to the bound w.h.p.

\begin{definition}\label{def:HkAB}
 	Graph $H_{k, \Delta}(A,B)$

	Let $V$ be a set of $n$ vertices, with $A \subset V$ an arbitrary subset satisfying $\floor{\frac{n}{4}} \leq |A| \leq \ceil{\frac{3n}{4}}$. % TODO: Reconsile floors and ceilings on bounds, want that B_0 > 3n/4, floor n/4 cieling 3n/4?
	Let B denote $V \setminus A$. The positive integers $k = \mathcal{O}\left(\frac{\log n}{\log \log n}\right)$ and $\Delta = \mathcal{O}(\sqrt{n})$ are fixed. We construct the graph $H_{k, \Delta}(A,B)$ on the vertex set $V$ using the following steps

	% TODO: Can we adjust k and delta slightly to work for k-regualr ends? 

	\textit{Step 1.} Let $\{S_i, 0 \leq i \leq k\}$ denote an arbitrary set of disjoint subsets of nodes in $V$ such that $|S_i| = \Delta$ for all $i$, $S_0 \subseteq A$, and $S_i \subseteq B$ for $1 \leq i \leq k$. We connect each node of $S_i$ to all nodes in $S_{i+1}$ for $0 \leq i \leq k - 1$. This generates a string of $k$ complete bipartite graphs, as seen in Figure \ref{fig:HkAB_1}.

	%TODO: Graph after step 1 

    \begin{figure}[h]
        \centering
        \begin{tikzpicture}[
    roundnode/.style={circle, fill=black, minimum size=0.5mm},
    align=center,
    node distance=0.5cm and 2cm
    ]
    %Nodes
    \node[roundnode] (1A) {};
    \node[roundnode] (1B) [below=of 1A] {};
    \node[roundnode] (1C) [below=of 1B] {};
    \node[roundnode] (1D) [below=of 1C] {};
    
    \node[draw=blue!20,inner sep=2mm,label=above:$S_0$,fit=(1A) (1A) (1A) (1D)] (S0fit) {};

    \node[draw=red!20,inner sep=6mm,label=below:$A$,fit=(S0fit)] {};
    
    \node[roundnode] (2A) [right=of 1A]{};
    \node[roundnode] (2B) [below=of 2A] {};
    \node[roundnode] (2C) [below=of 2B] {};
    \node[roundnode] (2D) [below=of 2C] {};
    
    \node[draw=blue!20,inner sep=2mm,label=above:$S_1$,fit=(2A) (2A) (2A) (2D)] (S1fit) {};
    
    \node[roundnode] (3A) [right=of 2A]{};
    \node[roundnode] (3B) [below=of 3A] {};
    \node[roundnode] (3C) [below=of 3B] {};
    \node[roundnode] (3D) [below=of 3C] {};
    
    \node[draw=blue!20,inner sep=2mm,label=above:$S_2$,fit=(3A) (3A) (3A) (3D)] {};
    
    \node[roundnode] (4A) [right=of 3A]{};
    \node[roundnode] (4B) [below=of 4A] {};
    \node[roundnode] (4C) [below=of 4B] {};
    \node[roundnode] (4D) [below=of 4C] {};
    
    \path (3A) -- node[auto=false]{\ldots} (4A);
    \path (3B) -- node[auto=false]{\ldots} (4B);
    \path (3C) -- node[auto=false]{\ldots} (4C);
    \path (3D) -- node[auto=false]{\ldots} (4D);
    
    \node[draw=blue!20,inner sep=2mm,label=above:$S_{k-1}$,fit=(4A) (4A) (4A) (4D)] {};
    
    \node[roundnode] (5A) [right=of 4A]{};
    \node[roundnode] (5B) [below=of 5A] {};
    \node[roundnode] (5C) [below=of 5B] {};
    \node[roundnode] (5D) [below=of 5C] {};
    
    \node[draw=blue!20,inner sep=2mm,label=above:$S_k$,fit=(5A) (5A) (5A) (5D)] (Skfit) {};
    
    \node[draw=red!20,inner sep=6mm,label=below:$B$,fit=(S1fit) (Skfit)] {};

    %Lines
    \draw[-] (1A) -- (2A);
    \draw[-] (1B) -- (2A);
    \draw[-] (1C) -- (2A);
    \draw[-] (1D) -- (2A);
    \draw[-] (1A) -- (2B);
    \draw[-] (1B) -- (2B);
    \draw[-] (1C) -- (2B);
    \draw[-] (1D) -- (2B);
    \draw[-] (1A) -- (2C);
    \draw[-] (1B) -- (2C);
    \draw[-] (1C) -- (2C);
    \draw[-] (1D) -- (2C);
    \draw[-] (1A) -- (2D);
    \draw[-] (1B) -- (2D);
    \draw[-] (1C) -- (2D);
    \draw[-] (1D) -- (2D);
    
    
    \draw[-] (2A) -- (3A);
    \draw[-] (2B) -- (3A);
    \draw[-] (2C) -- (3A);
    \draw[-] (2D) -- (3A);
    \draw[-] (2A) -- (3B);
    \draw[-] (2B) -- (3B);
    \draw[-] (2C) -- (3B);
    \draw[-] (2D) -- (3B);
    \draw[-] (2A) -- (3C);
    \draw[-] (2B) -- (3C);
    \draw[-] (2C) -- (3C);
    \draw[-] (2D) -- (3C);
    \draw[-] (2A) -- (3D);
    \draw[-] (2B) -- (3D);
    \draw[-] (2C) -- (3D);
    \draw[-] (2D) -- (3D);
    
    \draw[-] (4A) -- (5A);
    \draw[-] (4B) -- (5A);
    \draw[-] (4C) -- (5A);
    \draw[-] (4D) -- (5A);
    \draw[-] (4A) -- (5B);
    \draw[-] (4B) -- (5B);
    \draw[-] (4C) -- (5B);
    \draw[-] (4D) -- (5B);
    \draw[-] (4A) -- (5C);
    \draw[-] (4B) -- (5C);
    \draw[-] (4C) -- (5C);
    \draw[-] (4D) -- (5C);
    \draw[-] (4A) -- (5D);
    \draw[-] (4B) -- (5D);
    \draw[-] (4C) -- (5D);
    \draw[-] (4D) -- (5D);

    \end{tikzpicture}
        \caption{Construction of $H_{k, \Delta}(A,B)$ after Step 1}
        \label{fig:HkAB_1}
    \end{figure}
    
	\textit{Step 2.} Let $G_A$ be an arbitrary m-regular % TODO: Fill in once decided on ending graphs
	graph on $A \setminus S_0$ such that $\Phi(G_A) = \Theta(1)$. %TODO: remove this??

	% TODO: m-regular graph 
	% https://sites.math.rutgers.edu/~sk1233/courses/topics-S13/lec8.pdf
	% Margulis–Gabber–Galil convert to simple graph?

	We now connect each node of $S_0$ to $\Delta$ distinct nodes of $G_A$. First, we arbitrarily enumerate the nodes in $S_0$ and $G_A$ as $v_1, v_2, \dots, v_\Delta$ and $u_1, u_2, \dots, u_{|G_A|}$ respectively. Connect $v_1$ to the first $\Delta$ nodes in the $G_A$, starting with $u_1$. Connect $v_2$ to the next $\Delta$ nodes in $G_A$, starting with $u_{\Delta + 1}$.
	Repeat this process for each $v_i$, as in Figure \ref{fig:HkAB_2}.
	If there are more new connections than nodes in $G_A$, reuse nodes in $G_A$ starting again from $u_1$ and forming new connections in the enumerated order, i.e the $i^\text{th}$ new connection should be made with node $u_m$ where $m = i\mod |G_A|$.

    \begin{figure}[h]
        \centering
        \begin{tikzpicture}[
    roundnode/.style={circle, fill=black, minimum size=0.5mm},
    align=center,
    node distance=0.5cm and 2cm,
    redline/.style={-, red!40},
    blueline/.style={-, blue!40},
    greenline/.style={-, green!40},
    line width = 0.4mm
    ]
    %Nodes
    \node[roundnode, label=west:$u_1$] (1A) {};
    \node[roundnode, label=west:$u_2$] (1B) [below=of 1A] {};
    \node[roundnode, label=west:$u_\Delta$] (1C) [below=of 1B] {};
    \node[roundnode, label=west:$u_{\Delta+1}$] (1D) [below=of 1C] {};
    \node[roundnode] (1E) [below=of 1D] {};

    \node[inner sep=3mm,label=above:$G_A$,fit=(1A) (1A) (1A) (1E)] {};
    
    \node[roundnode, label=east:$v_1$, red!70] (2A) [right=of 1A]{};
    \node[roundnode, label=east:$v_2$, blue!70] (2B) [below=of 2A] {};
    \node[roundnode, label=east:$v_\Delta$, green!70] (2C) [below=of 2B] {};
    
    \node[inner sep=3mm,label=above:$S_0$,fit=(2A) (2A) (2A) (2C)] {};
    
    %Lines
    \draw[redline] (1A) -- (2A);
    \draw[redline] (1B) -- (2A);
    \draw[redline] (1C) -- (2A);

    \draw[blueline] (1D) -- (2B);
    \draw[blueline] (1E) -- (2B);
    \draw[blueline] (1A) -- (2B);

    \draw[greenline] (1B) -- (2C);
    \draw[greenline] (1C) -- (2C);
    \draw[greenline] (1D) -- (2C);

\end{tikzpicture}
        \caption{Illustration of the enumeration technique used in Step 2}
        \label{fig:HkAB_2}
    \end{figure}


	\textit{Step 3.} Let $G_B$ be an arbitrary m-regular graph on $B \setminus \bigcup_{i=1}^k S_i$ such that $\Phi(G_B) = \Theta(1)$. 
	We generate $G_B$ using the same construction as in Step 2. We then connect each node of $S_k$ to $\Delta$ distinct nodes of $G_B$ using the same enumeration method as in Step 2. This yields the graph $H_{k, \Delta}(A,B)$ illustrated in Figure \ref{fig:HkAB_3}.

    \begin{figure}[h]
        \centering
        \begin{tikzpicture}[
    roundnode/.style={circle, fill=black, minimum size=0.5mm},
    align=center,
    node distance=0.5cm and 2cm
    ]
    %Nodes
    \node[roundnode] (1A) {};
    \node[roundnode] (1B) [below=of 1A] {};
    \node[roundnode] (1C) [below=of 1B] {};
    \node[roundnode] (1D) [below=of 1C] {};
    \node[roundnode] (1E) [below=of 1D] {};

    \node[draw=blue!20,inner sep=2mm,label=above:$G_A$,fit=(1A) (1A) (1A) (1E)] {};
    
    \node[roundnode] (2A) [right=of 1A]{};
    \node[roundnode] (2B) [below=of 2A] {};
    \node[roundnode] (2C) [below=of 2B] {};
    \node[roundnode] (2D) [below=of 2C] {};
    
    \node[draw=blue!20,inner sep=2mm,label=above:$S_0$,fit=(2A) (2A) (2A) (2D)] {};
    
    \node[roundnode] (3A) [right=of 2A]{};
    \node[roundnode] (3B) [below=of 3A] {};
    \node[roundnode] (3C) [below=of 3B] {};
    \node[roundnode] (3D) [below=of 3C] {};
    
    \node[draw=blue!20,inner sep=2mm,label=above:$S_1$,fit=(3A) (3A) (3A) (3D)] {};
    
    \node[roundnode] (4A) [right=of 3A]{};
    \node[roundnode] (4B) [below=of 4A] {};
    \node[roundnode] (4C) [below=of 4B] {};
    \node[roundnode] (4D) [below=of 4C] {};
    
    \path (3A) -- node[auto=false]{\ldots} (4A);
    \path (3B) -- node[auto=false]{\ldots} (4B);
    \path (3C) -- node[auto=false]{\ldots} (4C);
    \path (3D) -- node[auto=false]{\ldots} (4D);
    
    \node[draw=blue!20,inner sep=2mm,label=above:$S_k$,fit=(4A) (4A) (4A) (4D)] {};
    
    \node[roundnode] (5A) [right=of 4A]{};
    \node[roundnode] (5B) [below=of 5A] {};
    \node[roundnode] (5C) [below=of 5B] {};
    \node[roundnode] (5D) [below=of 5C] {};
    \node[roundnode] (5E) [below=of 5D] {};
    \node[roundnode] (5F) [below=of 5E] {};
    
    \node[draw=blue!20,inner sep=2mm,label=above:$G_B$,fit=(5A) (5A) (5A) (5F)] {};
    
    
    %Lines
    \draw[-] (1A) -- (2A);
    \draw[-] (1B) -- (2A);
    \draw[-] (1C) -- (2A);
    \draw[-] (1D) -- (2A);

    \draw[-] (1A) -- (2B);
    \draw[-] (1B) -- (2B);
    \draw[-] (1C) -- (2B);
    \draw[-] (1E) -- (2B);

    \draw[-] (1A) -- (2C);
    \draw[-] (1B) -- (2C);
    \draw[-] (1D) -- (2C);
    \draw[-] (1E) -- (2C);

    \draw[-] (1A) -- (2D);
    \draw[-] (1C) -- (2D);
    \draw[-] (1D) -- (2D);
    \draw[-] (1E) -- (2D);
    
    \draw[-] (2A) -- (3A);
    \draw[-] (2B) -- (3A);
    \draw[-] (2C) -- (3A);
    \draw[-] (2D) -- (3A);
    \draw[-] (2A) -- (3B);
    \draw[-] (2B) -- (3B);
    \draw[-] (2C) -- (3B);
    \draw[-] (2D) -- (3B);
    \draw[-] (2A) -- (3C);
    \draw[-] (2B) -- (3C);
    \draw[-] (2C) -- (3C);
    \draw[-] (2D) -- (3C);
    \draw[-] (2A) -- (3D);
    \draw[-] (2B) -- (3D);
    \draw[-] (2C) -- (3D);
    \draw[-] (2D) -- (3D);
    
    \draw[-] (4A) -- (5A);
    \draw[-] (4A) -- (5B);
    \draw[-] (4A) -- (5C);
    \draw[-] (4A) -- (5D);

    \draw[-] (4B) -- (5E);
    \draw[-] (4B) -- (5F);
    \draw[-] (4B) -- (5A);
    \draw[-] (4B) -- (5B);

    \draw[-] (4C) -- (5C);
    \draw[-] (4C) -- (5D);
    \draw[-] (4C) -- (5E);
    \draw[-] (4C) -- (5F);

    \draw[-] (4D) -- (5A);
    \draw[-] (4D) -- (5B);
    \draw[-] (4D) -- (5C);
    \draw[-] (4D) -- (5D);

    \end{tikzpicture}
        \caption{Construction of $H_{k, \Delta}(A,B)$ after Step 3}
        \label{fig:HkAB_3}
    \end{figure}

	%TODO: explain how Winding around the ring increases the degree by at most an additive constant O(sqrt(n))^2 extra edges O(n) nodes in A \ S_0 so by O notation at most addivie constant difference

\end{definition}

We note that Definition \ref{def:HkAB} technically specifies a family of graphs, since the construction relies on choosing arbitrary sets of vertices, each of which yields a distinct graph. However, for the purposes of this analysis we think of $H_{k, \Delta}(A,B)$ as a single graph as all graphs in the family have the same structural properties required for the proof, so we can choose any of them. % TODO: Remove this paragraph? premature?

Using $H_{k, \Delta}(A,B)$, we now construct the following dynamic network for which Theorem \ref{theorem:AsyncUpperBound} is tight.

\begin{definition}
	$\rho$-diligent Dynamic Network
	
	First we define the sequences of node subsets $(A_t)_{t\geq 0}$ and $(B_t)_{t\geq0}$ with $A_t, B_t \subseteq V$ for all $t$. For each $t$, we construct $A_t$ and $B_t$ such that they partition $V$ into two subsets, as follows:

	Let $A_0$ be an arbitrary subset of $V$ of size $\floor{\frac{n}{4}}$ containing the first node to be aware of the rumour. % TODO: Any problem with ceiling function?
	Set $B_0 = V \setminus A_0$. Let $B_{t+1} = B_t \setminus I_{t+1}$, i.e. the set of nodes in $B_t$ that were still not informed of the rumour by round $t+1$. % TODO: Have we defined what I_t is?
	Let $A_{t+1} = V \setminus A_t$, i.e. the set of nodes that were either in $A_0$ or informed of the rumour by round $t+1$.

	Now we can define the dynamic network $\rho$-diligent Dynamic Network $\mathcal{G}(n, \rho) = (G_t)_{t\geq 0}$ itself. 
	% TODO: Check sequnce notation consistent, 0 not in N
	% TODO: Does \Delta satisfy conditions required for H_\Delta(A,B)
	% TODO: Standardise notation for H_K(A,B)

	Let $\Delta = \ceil{\frac{1}{\rho}}$, and $k = \mathcal{O}(\log n / \log \log n)$.
	Let $G_t = H_{k, \Delta}(A_t, B_t)$ while $|B_t| \geq \frac{n}{4}$. Note that since $\Delta = \mathcal{O}(\sqrt{n})$ by the definition of $H_{k, \Delta}(A,B)$, we must have that $\rho = \Omega\left(\frac{1}{\sqrt{n}}\right)$.
	Note also that $|A_t|$ is increasing in $t$ since $A_t$ is a superset of the informed nodes at time $t$, so there may be some time step $l$ where $|A_l| > \ceil{\frac{3n}{4}}$. Thus, for all time steps $t \geq l$, we cannot construct $H_{k, \Delta}(A_t,B_t)$. Hence, for all $t \geq l$, let $G_{t+1} = G_t$.
\end{definition}

In the following theorem, we use Theorem \ref{theorem:AsyncUpperBound} to bound the rumour spreading time of Algorithm \ref{NodeCentricAsyncAlgorithm} on a $\rho$-diligent dynamic network.

\begin{theorem}
	w.h.p, the rumour spreading time of Algorithm \ref{NodeCentricAsyncAlgorithm} on a $\rho$-diligent dynamic network with $n$ vertices, where $\rho = \Omega(\frac{1}{\sqrt{n}})$, is at most 
	$$
		\mathcal{O}\left(\left(\rho n + \frac{k}{\rho}\right)\log n\right)
	$$
\end{theorem}

\begin{proof}
	To determine a concrete bound using Theorem \ref{theorem:AsyncUpperBound}, we need to calculate $\Phi(G_t)$ and $\rho(G_t)$. Since $G_t = H_{k, \Delta}(A, B)$ for some $A$ and $B$ at each time step, it suffices to calculate the conductance and diligence for a general $H_{k, \Delta}(A, B)$.

	\textbf{Claim 1.} $\Phi(H_{k, \Delta}(A, B)) = \Theta\left(\frac{\Delta^2}{k\Delta^2 +n }\right)$

	For $q = 1, \dots, k$, let $A_q = G_A \cup \bigcup_{i=1}^q  S_i$. We want to calculate $\text{vol}(A_q)$. The degree of any arbitrary node in $S_i$ is $2\Delta$, since it is connected to all $\Delta$ nodes in $S_{i-1}$ and all $\Delta$ nodes in $S_{i+1}$, where we set $S_0 = G_A$ and $S_{k+1} = G_B$ for brevity. Since there are $\Delta$ nodes in each $S_i$, $\text{vol}(S_i) = 2\Delta^2$. By taking the disjoint union of $q$ $S_i$ subsets, we obtain that $\text{vol}(\bigcup_{i=1}^q) = 2q\Delta^2$. Now we consider $\text{vol}(G_A)$. Since $|A| = \Theta(n)$ and $|S_i| = \Delta = \mathcal{O}(\sqrt{n})$, 
	$$
		|G_A| = |A \setminus S_0| = |A| - |S_0| = \Theta(n) - \mathcal{O}(\sqrt{n}) = \Theta(n)
	$$
	By construction each node in $G_A$ is connected to exactly $m$ % TODO: Replace with actual number
	other nodes in $G_A$, so internal edges contribute $\Theta(n)$ degree to $\text{vol}(G_A)$. The only other edges which terminate in $G_A$ are the $\Delta^2$ edges between $G_A$ and $S_0$, thus $\text{vol}(G_A) = \Delta^2 + \Theta(n)$. By taking the disjoint union of $G_A$ and $\bigcup_{i=1}^q S_i$ we obtain $\text{vol}(A_q) = (2q + 1)\Delta^2 + \Theta(n)$. % TODO: Check as doesn't match paper.
	
	Note that for any $q$, $|E(A_q,\comp{A_q})| = \Delta^2$ since the only edges connecting $A_q$ and $\comp{A_q}$ are the edges passing between $S_q$ and $S_{q+1}$. This is most easily seen by inspecting Figure \ref{fig:HkAB_3}, and noting that $A_q$ corresponds to the first $q + 1$ sets of nodes in the string, while $\comp{A_q}$ corresponds to the remaining $k - q + 1$ sets of noes at the end of the string. By construction, each of these cuts consist of exactly $\Delta^2$ edges, thus $|E(A_q,\comp{A_q})| = \Delta^2$ as claimed.
	Hence, we can derive an upper bound on the conductance of $H_{k, \Delta}(A, B)$ by evaluating the conductance expression for the set $A_q$ instead of the set which achieves the minimum. 
	$$
		\Phi(H_{k, \Delta}(A, B)) \leq \frac{|E(A_q, \comp{A_q})|}{\min \left\{ \text{vol}(A_q), \text{vol}(\comp{A_q}) \right\} } % TODO: THIS INEQUALITY DOESN'T WORK
		\leq \frac{|E(A_q, \comp{A_q})|}{\text{vol}(A_q)} = \frac{\Delta^2}{(2q + 1)\Delta^2 + \Theta(n)}
	$$	
	% TODO: Prove that A_k is the minimal set

	% TODO: Finish this

	\textbf{Claim 2.} $\rho(H_{k, \Delta}(A,B)) = \Theta\left(\frac{1}{\Delta}\right)$

	First we consider the set of vertices $A = G_A \cup S_0$ for which the minimal diligence is achieved. % TODO: Check this works with A instead of A_1 (used in paper)
	We have that $\text{vol}(A) = 2\Delta^2 + \Theta(n)$ since $S_0$ contributes $2\Delta^2$ degree and $G_A$ contributes $\Theta(n)$ degree as shown in claim 1. Since $\Delta = \mathcal{O}(\sqrt{n})$, we obtain that 
	$$
		\text{vol}(A) = 2 \mathcal{O}(\sqrt{n})^2 + \Theta(n) = \mathcal{O}(n) + \Theta(n) = \Theta(n)
	$$ % TODO: Check adding theta and O gives theta
	Since $|A| = \Theta(n)$ by construction we have that the average degree of $A$ satisfies $$
		\bar{d}(A) = \frac{\text{vol}(A)}{|A|} = \frac{\Theta(n)}{\Theta(n)} = \Theta(1)
	$$
	To compute the diligence of $A$, we now consider the cut set $E(A, \comp{A})$. Since every endpoint of the cut set has degree $2\Delta$, we get that 
	$$
		\min_{\{u, v\} \in E(A, \comp{A})} \max \left\{ \frac{1}{d_u}, \frac{1}{d_v} \right\} = \frac{1}{2\Delta}
	$$
	Hence, the diligence of $A$ is 
	$$
		\rho(A) = \frac{\bar{d}(A)}{2\Delta} = \Theta\left(\frac{1}{\Delta}\right)
	$$ 
	
	Now we prove that no set with a smaller diligence exists. First note that we increase the degree of each node in $G_A$ by $\Theta(1)$ when joining them to $S_0$ and $S_1$ in Step 2 of the construction of $H_{k, \Delta}(A,B)$. To see this, we observe that the enumeration technique illustrated in Figure \ref{fig:HkAB_2} spreads the $\Delta^2$ additional edges evenly across $G_A$. Thus, the degree increase for an arbitrary node in $G_A$ is at most
	$$
		\ceil{\frac{\Delta^2}{|G_A|}}=\frac{\mathcal{O}(n)}{\Theta(n)} = \mathcal{O}(1)
	$$
	since $\Delta = \mathcal{O}(\sqrt{n})$ and $|G_A| = \Theta(n)$. 
	
	Since by construction, the internal edges in $G_A$ contribute $\Theta(1)$ degree to each node, we have that the degree for an arbitrary node in $G_A$ is $\Theta(1) + \mathcal{O}(1) = \Theta(1)$. As $G_B$ is connected to the string of bipartite graphs in an identical fashion during Step 2, all nodes of $G_B$ also have $\Theta(1)$ degree.  
	
	Let $S \subseteq V$ be an arbitrary subset of nodes. Since we have that all nodes in $G_A$ and $G_B$ have degree $\Theta(1)$, the maximum degree of any node in the graph is $2\Delta$. Hence 

	$$
	\min_{\{u, v\} \in E(S, \comp{S}) } \left\{ \max \left\{ \frac{1}{d_u},\frac{1}{d_v} \right\} \right\} \geq \frac{1}{2\Delta}
	$$
	Since all nodes have at least constant degree, we have that $\comp{d}(S) = \Omega(1)$. Thus 
	$$
		\rho(S) 
		= \comp{d}(S) \min_{\{u, v\} \in E(S, \comp{S}) } \left\{ \max \left\{ \frac{1}{d_u},\frac{1}{d_v} \right\} \right\}
		\geq \frac{\Omega(1)}{2\Delta}
		= \Omega\left(\frac{1}{\Delta}\right)
	$$
	Thus, since no subset of nodes exists with a lower diligence than the set $A$, our claim holds.

	We can now apply the bound from Theorem \ref{theorem:AsyncUpperBound}, to yield an upper bound on the spreading time of a $\rho$-diligent dynamic network $\mathcal{G}(n, \rho) = (G_t)_{t\geq 0}$. 
	
	From Theorem \ref{theorem:AsyncUpperBound}, we have that the first $T$ for which $\sum_t^T \Phi(G_t)\rho(G_t)$ exceeds $C \log n = \mathcal{O}(\log n)$ is an upper bound on the spreading time w.h.p.  Note that 
	$$
		\sum_t^T \Phi(G_t)\rho(G_t)
		= 
		T \Theta\left(\frac{\Delta^2}{k\Delta^2 +n }\right) \Theta\left(\frac{1}{\Delta}\right)
		= 
		T \Theta\left(\frac{\Delta}{k\Delta^2 +n }\right)
		= 
		T \Theta\left(\frac{1}{n \rho + \frac{k}{\rho}}\right)
	$$
	since $G_t = H_{k, \Delta}(A_t,B_t)$ for all $t$, where $\Delta = \ceil{\frac{1}{\rho}} = \Theta\left(\frac{1}{\rho}\right)$ by the definition of the dynamic network. Hence 
	$$
		\min \left\{T : \sum_t^T \Phi(G_t)\rho(G_t) \geq C \log n \right\}
		=
		\frac{\mathcal{O}(\log n)}{\Theta\left(\frac{1}{n \rho + \frac{k}{\rho}}\right)}
		= 
		\mathcal{O}\left(\log n \left(n \rho + \frac{k}{\rho}\right)\right)
	$$
	% TODO: Resolve k -> 0 log n -> +inf -> (check multaplicitve difference between log n from upper bound and 1/k for lower bound)
	% 		- why choice of log n / log log n for k?
\end{proof}

Now we verify that the upper bound is tight by computing a lower bound on the spreading time for a $\rho$-diligent dynamic network that holds w.h.p.

First we prove a lemma about the spread of the rumour down the chain of bipartite subgraphs in $H_{k, \Delta}(A,B)$.

\begin{lemma}\label{lemma:H_k,DeltaABOneStep}
	Suppose we spread a rumour according to Algorithm \ref{NodeCentricAsyncAlgorithm} on $G = H_{k, \Delta}(A,B)$ for some $A, B$ with $\Delta = \ceil{\frac{1}{\rho}}$. Assume that every node in $S_0$ is aware of the rumour, and no node in $B$ is informed. Then the probability that at least one node of $S_k$ is informed after a unit length of time is at most
	$$
		\frac{2^k}{k!}\ceil{\frac{1}{\rho}}
	$$	
\end{lemma}

\begin{proof}
	% TODO: Introduce two-push algorithm
	Let $I_i(\gamma)$ denote the number of informed nodes in $S_i$ at time $\gamma \in [0,1]$. 
	% Note that, by assumption, we have that $I_0(\gamma) = |S_0| = \Delta$ for all $\gamma \in [0,1]$. 

	We prove the claim by induction on $m$. % TODO: Change "claim" wording to be more specific - talking about expectation

	\underline{Base Case.} $\mathbb{E}I_1(\tau) \leq 2 \tau \Delta$

	Since each node in $S_0$ is aware of the rumour, each of the $\Delta^2$ edges from $S_0$ to $S_1$ is pushing the rumour forward at rate $\frac{2}{\Delta}$ for all times $\gamma \in [0,1]$. Thus, by the superposition property of Poisson processes, the overall rate at which the rumour is pushed from $S_0$ to any node of $S_1$ is $\frac{2}{\Delta}\Delta^2 = 2\Delta$. Let $N_1(\tau)$ denote the number of times any node of $S_1$ is informed of the rumour in the interval $[0,\tau]$ (even if it already knew the rumour). By the theory of Poisson processes, $N_1(\tau)$ is distributed according to a Poisson distribution with mean $2\tau\Delta$. Note $I_i(\tau) \leq N_1(\tau)$ since the same nodes of $S_1$ may be contacted multiple times. Hence $\mathbb{E}I_1(\tau) \leq \mathbb{E}N_1(\tau) = 2\tau\Delta$ as required.
	
	\underline{Inductive Case.} $\mathbb{E}I_k(\tau) \leq \frac{2^k \tau^k}{k!}\Delta$

	Assume that for all $m \leq k - 1$ and $\gamma \in [0,1]$, $\mathbb{E}I_m(\gamma) \leq \frac{2^m \gamma^m}{m!}\Delta$ holds.
	By the tower law,
	$$
		\mathbb{E}I_k(\tau) = \mathbb{E}\mathbb{E}[I_k(\tau) |I_{k-1}(\gamma), \gamma \in [0, \tau]] % TODO: [0, \tau] or [0,1] like in the paper
	$$
	As in the base case, let $N_k(\tau)$ denote the number of times a node in $S_k$ is informed of the rumour in the interval $[0, \tau]$. Since $I_k(\tau) \leq N_k(\tau)$,
	$$
		\mathbb{E}[I_k(\tau) |I_{k-1}(\gamma), \gamma \in [0, 1]] \leq \mathbb{E}[N_k(\tau) |I_{k-1}(\gamma), \gamma \in [0, \tau]]
	$$
	At time $\gamma \in [0,1]$, there are $I_{k-1}(\gamma)$ nodes aware of the rumour in $S_{k-1}$, and pushing the rumour down each of their $\Delta$ edges to $S_k$. Since each of these edges pushes the rumour according to a Poisson process of rate $\frac{2}{\Delta}$, by the superposition of Poisson processes the nodes of $S_{k-1}$ inform the nodes of $S_k$ according to a Poisson processes of rate $\frac{2}{\Delta}I_{k-1}(\gamma)\Delta = 2 I_{k-1}(\gamma)$.
	Thus, $N_k(\tau)|(I_{k-1}(\gamma), \gamma \in [0, \tau])$ is an inhomogeneous Poisson process, where the value of the rate function at time $\gamma$ is $2 I_{k-1}(\gamma)$. By % TODO: CITE HOMOGENEOUS THEOREM
	, 
	\begin{align*}
		\mathbb{E}[N_k(\tau)|I_{k-1}(\gamma), \gamma \in [0, \tau]] &= \mathbb{E}\int_0^\tau 2 I_{k-1}(\gamma) d\gamma \\
		&= 2 \int_0^\tau \mathbb{E}I_{k-1}(\gamma) d\gamma \\ % TODO: How can we swap integral and expectation? Fubinis theorem since I_{k-1}(\gamma) \leq \Delta
		& \leq 2 \int_0^\tau \frac{2^{k-1} \gamma^{k-1}}{(k-1)!} \Delta d\gamma \\
		\intertext{by the induction hypothesis}
		& = \frac{2^k \tau^k}{k!} \Delta
	\end{align*}
	Hence 
	$$
		\mathbb{E}I_k(\tau) \leq \mathbb{E}\frac{2^k \tau^k}{k!}\Delta = \frac{2^k \tau^k}{k!}\Delta
	$$

	% TODO: Prove choice of k and wrap up
\end{proof}


% TODO: Talk about restrictions on rho and k later?
\begin{theorem}
	w.h.p, the rumour spreading time of Algorithm \ref{NodeCentricAsyncAlgorithm} on a $\rho$-diligent dynamic network with $n$ vertices, where $\rho = \Omega(\frac{1}{\sqrt{n}})$, is at least 
	$$
		\Omega\left(\frac{n \rho}{k}\right)
	$$
\end{theorem}

% TODO: Interesting dependence on diligence -> higher diligence = longer to spread?

\begin{proof} 
	Let $T_0$ be the first time that all the nodes of $S_0$ are aware of the rumour. To simplify our analysis, suppose that we are at time $T_0$, hence for all remaining time steps we can assume that every node in $S_0$ is aware of the rumour. % TODO: Check this - we choose S_0 artibitratily, instead consider first time all nodes of A are aware of the rumour? (changes size of B?)

	By the definition of a $\rho$-diligent network, at time $t=0$  we have that all the informed nodes are located in $A_0$, and no node of $B_0$ knows the rumour. Thus, by Lemma \ref{lemma:H_k,DeltaABOneStep}, w.h.p at time $t = 1$, no node of $S_k$ is informed of the rumour. Since the only way the rumour can reach the node set $G_B = B \setminus \bigcup_{i=1}^k S_i$ is through $S_k$ (see Figure \ref{fig:HkAB_3}), we can conclude that w.h.p after the first time step no node of $G_B$ is aware of the rumour. Equivalently we can say that the set of nodes $B_0 \cap I_1 \subseteq \bigcup_i^k S_i$, i.e. all nodes of $B_0$ that are aware of the rumour after one time step are confined to the set of $S_i$s in $G_1$. Hence, 
	\begin{equation}\label{eq:firstStepinfectedNodesOfB}
		|B_0 \cap I_1| \leq \left|\bigcup_{i=1}^k S_i\right| = k\Delta
	\end{equation}
	since each of the $k$ disjoint node sets $S_i$ have size $\Delta$. Since $B_1 = B_0 \setminus I_1$
	$$
		|B_1| = |B_0 \setminus I_1| = |B_0| - |B_0 \cap I_1| \geq \frac{3n}{4} - k\Delta
	$$ % TODO: Corrected typo above
	where the final inequality holds due to inequality (\ref{eq:firstStepinfectedNodesOfB}) and the fact that $|B_0| \geq \frac{3n}{4}$. 
	% TODO: Mention S, G_A, G_B different each timestep - chosen arbitrarily

	We now repeat this analysis for the first $\frac{n}{4k\Delta}$ time steps, by first showing that w.h.p no node in $S_k$ is made aware of the rumour in the first $\frac{n}{4k\Delta}$ time steps. Let $\mathcal{A}_t$ denote the event that in time step $t$ some node in $S_k$ is made aware of the rumour. Note that the event 
	$$\mathcal{A} = \left\{S_k \text{ is informed of the rumour any time step } t \leq \frac{n}{4k\Delta} \right\} = \bigcup_{t=1}^\frac{n}{4k\Delta} \mathcal{A}_t
	$$
	hence by the union bound and Lemma \ref{lemma:H_k,DeltaABOneStep},
	$$
		\mathbb{P}(\mathcal{A}) 
		\leq \sum_{t=1}^\frac{n}{4k\Delta} \mathbb{P}(A_t) 
		\leq \frac{n}{4k\Delta}n^{-c}
		= \frac{n^{1-c}}{4k\Delta} \leq n^{1-c}
	$$
	as $k, \Delta \geq 1$. Since $c > 1$ % TODO: Introduce this restriction earlier
	we have that w.h.p in the first $\frac{n}{4k\Delta}$ time steps the rumour does not reach $S_k$. 

	Thus, for $m \leq \frac{n}{4k\Delta}$ we can repeat the previous analysis inductively as follows % TODO: Rephrase this
	$$
		|B_m| = |B_{m-1} \setminus I_m| \geq \frac{3n}{4} - (m-1)k\Delta - k\Delta = \frac{3n}{4} - mk\Delta
	$$
	Note that for $m = \frac{n}{4k\Delta}$, we get that 
	$$
		|B_m| \geq \frac{3n}{4} - \frac{nk\Delta}{4k\Delta} = \frac{n}{2}
	$$
	Thus, w.h.p after $\frac{n}{4k\Delta}$ time steps, there are still at least $\frac{n}{2}$ nodes who are not aware of the rumour (since no node in $B_i$ is aware of the rumour at the beginning of each time step by construction). Hence, we can conclude the algorithm takes at least 
	$$
		\Omega\left(\frac{n}{4k\Delta}\right) = \Omega\left(\frac{n\rho}{k}\right)
	$$
	time to spread the rumour to all nodes of $\mathcal{G}(n, \rho)$, where $\Delta = \ceil{\frac{1}{\rho}}$.
\end{proof}

\begin{theorem}
	For all large $n$ and $\rho = \Omega\left(\frac{1}{\sqrt{n}}\right)$, there exists a dynamic network on $n$ vertices with diligence $\rho$ for which the bound holds up to a factor of 
	$$
		\mathcal{O}\left(\frac{\log n}{\log \log n}\right)
	$$	
\end{theorem}

% TODO: Significance of over all \rho - can't tighten bound if diligence over \rho