\section{Proof that the asynchronous rumour spreading time bound is almost tight}
\label{section:AsyncLowerBound}

In this section, we prove that the upper bound derived in Section \ref{AsyncUpperBoundSection} is tight, by constructing an example where the spread time is close to the bound with high probability.

\begin{definition}\label{def:HkAB}
 	Graph $H_k(A,B)$

	Let $V$ be a set of $n$ vertices, with $A \subset V$ an arbitrary subset satisfying $\frac{n}{4} \leq |A| \leq \frac{3n}{4}$. Let B denote $V \setminus A$. Let $k = \mathcal{O}\left(\frac{\log n}{\log \log n}\right)$ and $\Delta = \mathcal{O}(\sqrt{n})$ be arbitrary positive integers. We construct the graph $H_k(A,B)$ on the vertex set $V$ using the following steps

	% TODO: Can we adjust k and delta slightly to work for k-regualr ends? 

	\textit{Step 1.} Let $\{S_i, 0 \leq i \leq k\}$ denote an arbitrary set of disjoint subsets of nodes in $V$ such that $|S_i| = \Delta$ for all $i$, $S_0 \subseteq A$ and $S_i \subseteq B$ for $1 \leq i \leq k$. We connect each node of $S_i$ to all nodes in $S_{i+1}$ for $0 \leq i \leq k - 1$. This generates a string of $k$ complete bipartite graphs, as seen in Figure \ref{fig:HkAB_1}.

	%TODO: Graph after step 1 

    \begin{figure}[h]
        \centering
        \begin{tikzpicture}[
    roundnode/.style={circle, fill=black, minimum size=0.5mm},
    align=center,
    node distance=0.5cm and 2cm,
    line width = 0.3mm
    ]
    %Nodes
    \node[roundnode] (1A) {};
    \node[roundnode] (1B) [below=of 1A] {};
    \node[roundnode] (1C) [below=of 1B] {};
    \node[roundnode] (1D) [below=of 1C] {};
    
    \node[draw=blue!20,inner sep=2mm,label=above:$S_0$,fit=(1A) (1A) (1A) (1D)] (S0fit) {};

    \node[draw=red!20,inner sep=6mm,label=below:$A$,fit=(S0fit)] {};
    
    \node[roundnode] (2A) [right=of 1A]{};
    \node[roundnode] (2B) [below=of 2A] {};
    \node[roundnode] (2C) [below=of 2B] {};
    \node[roundnode] (2D) [below=of 2C] {};
    
    \node[draw=blue!20,inner sep=2mm,label=above:$S_1$,fit=(2A) (2A) (2A) (2D)] (S1fit) {};
    
    \node[roundnode] (3A) [right=of 2A]{};
    \node[roundnode] (3B) [below=of 3A] {};
    \node[roundnode] (3C) [below=of 3B] {};
    \node[roundnode] (3D) [below=of 3C] {};
    
    \node[draw=blue!20,inner sep=2mm,label=above:$S_2$,fit=(3A) (3A) (3A) (3D)] {};
    
    \node[roundnode] (4A) [right=of 3A]{};
    \node[roundnode] (4B) [below=of 4A] {};
    \node[roundnode] (4C) [below=of 4B] {};
    \node[roundnode] (4D) [below=of 4C] {};
    
    \path (3A) -- node[auto=false]{\ldots} (4A);
    \path (3B) -- node[auto=false]{\ldots} (4B);
    \path (3C) -- node[auto=false]{\ldots} (4C);
    \path (3D) -- node[auto=false]{\ldots} (4D);
    
    \node[draw=blue!20,inner sep=2mm,label=above:$S_{k-1}$,fit=(4A) (4A) (4A) (4D)] {};
    
    \node[roundnode] (5A) [right=of 4A]{};
    \node[roundnode] (5B) [below=of 5A] {};
    \node[roundnode] (5C) [below=of 5B] {};
    \node[roundnode] (5D) [below=of 5C] {};
    
    \node[draw=blue!20,inner sep=2mm,label=above:$S_k$,fit=(5A) (5A) (5A) (5D)] (Skfit) {};
    
    \node[draw=red!20,inner sep=6mm,label=below:$B$,fit=(S1fit) (Skfit)] {};

    %Lines
    \draw[-] (1A) -- (2A);
    \draw[-] (1B) -- (2A);
    \draw[-] (1C) -- (2A);
    \draw[-] (1D) -- (2A);
    \draw[-] (1A) -- (2B);
    \draw[-] (1B) -- (2B);
    \draw[-] (1C) -- (2B);
    \draw[-] (1D) -- (2B);
    \draw[-] (1A) -- (2C);
    \draw[-] (1B) -- (2C);
    \draw[-] (1C) -- (2C);
    \draw[-] (1D) -- (2C);
    \draw[-] (1A) -- (2D);
    \draw[-] (1B) -- (2D);
    \draw[-] (1C) -- (2D);
    \draw[-] (1D) -- (2D);
    
    
    \draw[-] (2A) -- (3A);
    \draw[-] (2B) -- (3A);
    \draw[-] (2C) -- (3A);
    \draw[-] (2D) -- (3A);
    \draw[-] (2A) -- (3B);
    \draw[-] (2B) -- (3B);
    \draw[-] (2C) -- (3B);
    \draw[-] (2D) -- (3B);
    \draw[-] (2A) -- (3C);
    \draw[-] (2B) -- (3C);
    \draw[-] (2C) -- (3C);
    \draw[-] (2D) -- (3C);
    \draw[-] (2A) -- (3D);
    \draw[-] (2B) -- (3D);
    \draw[-] (2C) -- (3D);
    \draw[-] (2D) -- (3D);
    
    \draw[-] (4A) -- (5A);
    \draw[-] (4B) -- (5A);
    \draw[-] (4C) -- (5A);
    \draw[-] (4D) -- (5A);
    \draw[-] (4A) -- (5B);
    \draw[-] (4B) -- (5B);
    \draw[-] (4C) -- (5B);
    \draw[-] (4D) -- (5B);
    \draw[-] (4A) -- (5C);
    \draw[-] (4B) -- (5C);
    \draw[-] (4C) -- (5C);
    \draw[-] (4D) -- (5C);
    \draw[-] (4A) -- (5D);
    \draw[-] (4B) -- (5D);
    \draw[-] (4C) -- (5D);
    \draw[-] (4D) -- (5D);

    \end{tikzpicture}
        \caption{Construction of $H_k(A,B)$ after Step 1}
        \label{fig:HkAB_1}
    \end{figure}
    
	\textit{Step 2.} Let $G_A$ be an arbitrary m-regular %TODO: Fill in once decided on ending graphs
	graph on $A \setminus S_0$ such that $\Phi(G_A) = \Theta(1)$. %TODO: remove this??

	% TODO: m-regular graph 
	% https://sites.math.rutgers.edu/~sk1233/courses/topics-S13/lec8.pdf
	% Margulis–Gabber–Galil convert to simple graph?

	We now connect each node of $S_0$ to $\Delta$ distinct nodes of $G_A$. First, we arbitrarily enumerate the nodes in $S_0$ and $G_A$ as $v_1, v_2, \dots, v_\Delta$ and $u_1, u_2, \dots, u_{|G_A|}$ respectively. Connect $v_1$ to the first $\Delta$ nodes in the $G_A$, starting with $u_1$. Connect $v_2$ to the next $\Delta$ nodes in $G_A$, starting with $u_{\Delta + 1}$.
	Repeat this process for each $v_i$, as in Figure \ref{fig:HkAB_2}.
	If there are more new connections than nodes in $G_A$, reuse nodes in $G_A$ starting again from $u_1$ and forming new connections in the enumerated order, i.e the $i^\text{th}$ new connection should be made with node $u_m$ where $m = i\mod |G_A|$.

    \begin{figure}[h]
        \centering
        \begin{tikzpicture}[
    roundnode/.style={circle, fill=black, minimum size=0.5mm},
    align=center,
    node distance=0.5cm and 2cm,
    redline/.style={-, red},
    blueline/.style={-, blue},
    greenline/.style={-, green}
    ]
    %Nodes
    \node[roundnode, label=west:$u_1$] (1A) {};
    \node[roundnode, label=west:$u_2$] (1B) [below=of 1A] {};
    \node[roundnode, label=west:$u_\Delta$] (1C) [below=of 1B] {};
    \node[roundnode, label=west:$u_{\Delta+1}$] (1D) [below=of 1C] {};
    \node[roundnode] (1E) [below=of 1D] {};

    \node[inner sep=3mm,label=above:$G_A$,fit=(1A) (1A) (1A) (1E)] {};
    
    \node[roundnode, label=east:$v_1$, red] (2A) [right=of 1A]{};
    \node[roundnode, label=east:$v_2$, blue] (2B) [below=of 2A] {};
    \node[roundnode, label=east:$v_\Delta$, green] (2C) [below=of 2B] {};
    
    \node[inner sep=3mm,label=above:$S_0$,fit=(2A) (2A) (2A) (2C)] {};
    
    %Lines
    \draw[redline] (1A) -- (2A);
    \draw[redline] (1B) -- (2A);
    \draw[redline] (1C) -- (2A);

    \draw[blueline] (1D) -- (2B);
    \draw[blueline] (1E) -- (2B);
    \draw[blueline] (1A) -- (2B);

    \draw[greenline] (1B) -- (2C);
    \draw[greenline] (1C) -- (2C);
    \draw[greenline] (1D) -- (2C);

\end{tikzpicture}
        \caption{Illustration of the enumeration technique used in Step 2}
        \label{fig:HkAB_2}
    \end{figure}


	\textit{Step 3.} Let $G_B$ be an arbitrary m-regular graph on $B \setminus \bigcup_{i=1}^k S_i$ such that $\Phi(G_B) = \Theta(1)$. 
	We generate $G_B$ using the same construction as in Step 2. We then connect each node of $S_k$ to $\Delta$ distinct nodes of $G_B$ using the same enumeration method as in Step 2. This yields the graph $H_k(A,B)$ illustrated in Figure \ref{fig:HkAB_3}.

    \begin{figure}[h]
        \centering
        \begin{tikzpicture}[
    roundnode/.style={circle, fill=black, minimum size=0.5mm},
    align=center,
    node distance=0.5cm and 2cm,
    line width = 0.3mm
    ]
    %Nodes
    \node[roundnode] (1A) {};
    \node[roundnode] (1B) [below=of 1A] {};
    \node[roundnode] (1C) [below=of 1B] {};
    \node[roundnode] (1D) [below=of 1C] {};
    \node[roundnode] (1E) [below=of 1D] {};

    \node[draw=blue!20,inner sep=2mm,label=above:$G_A$,fit=(1A) (1A) (1A) (1E)] {};
    
    \node[roundnode] (2A) [right=of 1A]{};
    \node[roundnode] (2B) [below=of 2A] {};
    \node[roundnode] (2C) [below=of 2B] {};
    \node[roundnode] (2D) [below=of 2C] {};
    
    \node[draw=blue!20,inner sep=2mm,label=above:$S_0$,fit=(2A) (2A) (2A) (2D)] {};
    
    \node[roundnode] (3A) [right=of 2A]{};
    \node[roundnode] (3B) [below=of 3A] {};
    \node[roundnode] (3C) [below=of 3B] {};
    \node[roundnode] (3D) [below=of 3C] {};
    
    \node[draw=blue!20,inner sep=2mm,label=above:$S_1$,fit=(3A) (3A) (3A) (3D)] {};
    
    \node[roundnode] (4A) [right=of 3A]{};
    \node[roundnode] (4B) [below=of 4A] {};
    \node[roundnode] (4C) [below=of 4B] {};
    \node[roundnode] (4D) [below=of 4C] {};
    
    \path (3A) -- node[auto=false]{\ldots} (4A);
    \path (3B) -- node[auto=false]{\ldots} (4B);
    \path (3C) -- node[auto=false]{\ldots} (4C);
    \path (3D) -- node[auto=false]{\ldots} (4D);
    
    \node[draw=blue!20,inner sep=2mm,label=above:$S_k$,fit=(4A) (4A) (4A) (4D)] {};
    
    \node[roundnode] (5A) [right=of 4A]{};
    \node[roundnode] (5B) [below=of 5A] {};
    \node[roundnode] (5C) [below=of 5B] {};
    \node[roundnode] (5D) [below=of 5C] {};
    \node[roundnode] (5E) [below=of 5D] {};
    \node[roundnode] (5F) [below=of 5E] {};
    
    \node[draw=blue!20,inner sep=2mm,label=above:$G_B$,fit=(5A) (5A) (5A) (5F)] {};
    
    
    %Lines
    \draw[-] (1A) -- (2A);
    \draw[-] (1B) -- (2A);
    \draw[-] (1C) -- (2A);
    \draw[-] (1D) -- (2A);

    \draw[-] (1A) -- (2B);
    \draw[-] (1B) -- (2B);
    \draw[-] (1C) -- (2B);
    \draw[-] (1E) -- (2B);

    \draw[-] (1A) -- (2C);
    \draw[-] (1B) -- (2C);
    \draw[-] (1D) -- (2C);
    \draw[-] (1E) -- (2C);

    \draw[-] (1A) -- (2D);
    \draw[-] (1C) -- (2D);
    \draw[-] (1D) -- (2D);
    \draw[-] (1E) -- (2D);
    
    \draw[-] (2A) -- (3A);
    \draw[-] (2B) -- (3A);
    \draw[-] (2C) -- (3A);
    \draw[-] (2D) -- (3A);
    \draw[-] (2A) -- (3B);
    \draw[-] (2B) -- (3B);
    \draw[-] (2C) -- (3B);
    \draw[-] (2D) -- (3B);
    \draw[-] (2A) -- (3C);
    \draw[-] (2B) -- (3C);
    \draw[-] (2C) -- (3C);
    \draw[-] (2D) -- (3C);
    \draw[-] (2A) -- (3D);
    \draw[-] (2B) -- (3D);
    \draw[-] (2C) -- (3D);
    \draw[-] (2D) -- (3D);
    
    \draw[-] (4A) -- (5A);
    \draw[-] (4A) -- (5B);
    \draw[-] (4A) -- (5C);
    \draw[-] (4A) -- (5D);

    \draw[-] (4B) -- (5E);
    \draw[-] (4B) -- (5F);
    \draw[-] (4B) -- (5A);
    \draw[-] (4B) -- (5B);

    \draw[-] (4C) -- (5C);
    \draw[-] (4C) -- (5D);
    \draw[-] (4C) -- (5E);
    \draw[-] (4C) -- (5F);

    \draw[-] (4D) -- (5A);
    \draw[-] (4D) -- (5B);
    \draw[-] (4D) -- (5C);
    \draw[-] (4D) -- (5D);

    \end{tikzpicture}
        \caption{Construction of $H_k(A,B)$ after Step 3}
        \label{fig:HkAB_3}
    \end{figure}

	%TODO: explain how Winding around the ring increases the degree by at most an additive constant O(sqrt(n))^2 extra edges O(n) nodes in A \ S_0 so by O notation at most addivie constant difference

\end{definition}

We note that Definition \ref{def:HkAB} technically specifies a family of graphs, since the construction relies on choosing arbitrary permutations of vertices, each of which yields a distinct graph. However, for the purposes of this analysis we think of $H_k(A,B)$ as a single graph as all graphs in the family are isomorphic, and thus have the same structural properties needed for the proof. % TODO: Remove this paragraph? premature?

\begin{definition}
	$\rho$-diligent Dynamic Network
	
	First we define the sequences of node subsets $(A_t)_{t\geq 0}$ and $(B_t)_{t\geq0}$ with $A_t, B_t \subseteq V$ for all $t$. For each $t$, we construct $A_t$ and $B_t$ such that they partition $V$ into two subsets, as follows:

	Let $A_0$ be an arbitrary subset of $V$ of size $\ceil{\frac{n}{4}}$ containing the first node to be aware of the rumour. % TODO: Any problem with ceiling function?
	Set $B_0 = V \setminus A_0$. Let $B_{t+1} = B_t \setminus I_{t+1}$, i.e. the set of nodes in $B_t$ that were still not informed of the rumour by round $t+1$. Let $A_{t+1} = V \setminus A_t$, i.e. the set of nodes that were either in $A_0$ or informed of the rumour by round $t+1$.

	Now we can define the dynamic network $\rho$-diligent Dynamic Network $\mathcal{G}(n, \rho) = (G_t)_{t\geq 0}$ itself. 
	% TODO: Check sequnce notation consistent, 0 not in N
	% TODO: Does \Delta satisfy conditions required for H_\Delta(A,B)
	% TODO: Standardise notation for H_K(A,B)

	Let $\Delta = \ceil{\frac{1}{\rho}}$. Let $G_t = H_\Delta(A_t, B_t)$ while $|B_t| \geq \frac{n}{4}$. Note that $|B_t|$ is decreasing in $t$ since $B_t$ is a subset of the uniformed nodes at time $t$, so at some time step $l$ we have that $|B_l| < \frac{n}{4}$. For all time steps $t \geq l$, let $G_{t+1} = G_t$, since we cannot construct $H_\Delta(A,B)$ if $|B| < \frac{n}{4}$ as this would imply $|A| > \frac{3n}{4}$. % TODO: Reword ending$
\end{definition}