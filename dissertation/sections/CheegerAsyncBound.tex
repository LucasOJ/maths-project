\section{Bounding the asynchronous rumour spreading time in terms of the Cheeger constant}
\label{AsyncCheegerBound}

\subsection{Model}

TODO: Define asynchronous model - explain impact of changes in comparison to the other model (in this model each edge has rate 1 instead of 1/deg(v), bound not dependent on degrees of nodes)


Each edge is associated with a Poisson Process of unit rate. When the Possion Processes fires, if one of the endpoints knows the rumour then both endpoints immediately become aware of the rumour. 
% TODO: CHANGE WORDING - what if nodes know rumour already

\begin{definition}
	Edge-boundary of a set of vertices $A \subseteq V$
	$$
		\partial A = \left\{ \{u, v\} \in E \mid u \in A, v \in V \setminus A \right\} 
	$$
\end{definition}

\begin{definition}
	Cheeger Constant of a graph $G$
	$$
		h(G) = \min_{A \subset V, 0 < |A| \leq \frac{n}{2}} \frac{|\partial A|}{|A|}
	$$

\end{definition}

\begin{theorem}
	Let $\mathcal{G} = (G_t)_{t \in \mathbb{N}}$ be an $n$-node asychronus PUSH-model dynamic network, where at least one node is aware of a rumour in $G_0$.

	Then, w.h.p, the rumour will have spread to all the nodes of $\mathcal{G}$ in time at most

	$$
		\min \left\{t : \sum_{k=0}^t h(G_k) \geq C \log n \right\} 
	$$

	\noindent
	for a sufficiently large $C$
\end{theorem}

TODO: Explain how can apply the following result in the proof from the first asychronus model

\begin{proof}

	% TODO: Move tau' defn to lemma statement
	Let $\tau'$ be the first time at which the number of informed nodes increases by at least $\frac{m(\tau)}{2}$ 
	
	$$
		\tau' = \min\left\{\gamma : |I_{\gamma}| \geq |I_\tau| + \frac{m(\tau)}{2}\right\}
	$$
	Let $ \gamma \in [\tau, \tau')$.

	Suppose  $|I_\gamma| \leq \frac{n}{2}$. By the definition of the Cheeger constant we have that $h(G_\gamma) \leq \frac{|\partial I_\gamma|}{|I_\gamma|}$. Note that $ \partial I_\gamma = E(I_\gamma, U_\gamma)$, since $\partial I_\gamma$ is the set of edges with exactly one endpoint in $I_\gamma$, thus the other endpoint must be in the complement of $I_\gamma$, namely $U_\gamma$. Thus

	\begin{align*}
		\lambda(\gamma) &= |E(I_\gamma, U_\gamma)| \\
		& = |\partial I_\gamma| \\
		& \geq h(G_\gamma) |I_\gamma| \\
		& \geq h(G_\gamma) |I_\tau| & \text{since } |I_\gamma| \text{ is increasing} \\
		& \geq h(G_\gamma) \frac{m(\tau)}{2}
	\end{align*}
		
	%TODO: THIS JUST EXCLUDES THE ERROR CASE
	Now suppose $|I_\gamma| > \frac{n}{2}$, thus $|U_\gamma| < \frac{n}{2}$. Hence, by the definition of the Cheeger constant we have that $h(G_\gamma) \leq \frac{|\partial U_\gamma|}{|U_\gamma|}$. Note that $ \partial U_\gamma = E(I_\gamma, U_\gamma)$, since $\partial U_\gamma$ is the set of edges with exactly one endpoint in $U_\gamma$,  thus the other endpoint must be in the complement of $U_\gamma$, namely $I_\gamma$. Thus 

	\begin{align*}
		\lambda(\gamma) &= |E(I_\gamma, U_\gamma)| \\
		& = |\partial U_\gamma| \\
		& \geq h(G_\gamma) |U_\gamma| \\
	\end{align*} 

	Since $|I_t| + |U_t| = n$ for all times $t$, we can reformulate the definition of $\tau'$ as follows

	$$
		\tau' = \min \left\{ \gamma : |U_\tau| - \frac{m(\tau)}{2} \geq |U_\gamma| \right\} 
	$$ 
	
	Since $\gamma < \tau'$ we have that

	\begin{align*}
		|U_\gamma| & \geq |U_\tau| - \frac{m(\tau)}{2} \\
		& \geq |U_\tau| - \frac{|U_\tau|}{2} \\
		& = \frac{|U_\tau|}{2} \\
	\end{align*}
	
	Hence $\lambda(\gamma) \geq h(G_\gamma)\frac{|U_\tau|}{2} = h(G_\gamma)\frac{m(\tau)}{2}$

\end{proof}

\subsection{Applying the bound to Edge-Markovian evolution}

TODO
\begin{enumerate}
	\item Prove that an individual G(n,p) ER graph satisfies $h(G) > p$ w.h.p
	\item Prove that the first log(n)/p ER graphs satisfy this property w.h.p (union bound on event at least one not satisfying)- see edge-markovian proof
	\item Combine with result from previous bound to get O(log(n)/p) concrete bound
	\item If possible - prove spread time is $\Theta(\log(n)/p)$
\end{enumerate}

\subsection{Simulations}

TODO: Evaluate bound with simulatons if can't prove spread time is  $\Theta(\log(n)/p)$