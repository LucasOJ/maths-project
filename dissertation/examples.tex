\section{Formating}

We can \emph{emphasis} some words, i.e., make them \emph{italic}, and we can make some words \textbf{bold}.
Note how using a new line in the code does not correspond to a new line in the output file.
Same if we have        a           large                white                   space.

Instead, if we want a new line/new paragraph, you need to press enter twice, or use \\
which starts a new line but not a new paragraph.

\subsection{lists}

Lists can be numbered or ununmbered, and you can have sub-list inside a list.

\begin{enumerate}
	\item This is the first item in a numbered list.

	\item And the second
	
	\item 
	\begin{enumerate}
		\item Here the third item is in fact a numbered sub-list.
		\item item 2 of the numbered sub-list
	\end{enumerate}

	\item 
	\begin{itemize}
		\item Here the fourth item is an unnumbered sub-list.
		\item item 2 of the unnumbered sub-list
	\end{itemize}
\end{enumerate}

\subsection{Definitions and theorems}

Definitions, theorems, lemmas and so on, are 'enviroments' (like documents and lists). They need to begin and end.

\begin{definition}\label{my_def}
	A \emph{label} allows the user to tell Latex 'remember the numbering of that definition/theorem/equation'
\end{definition}

\begin{lemma} \label{my_lem}
	If something has a label, then we can refer to it, without knowing what number it is 
\end{lemma}

\begin{proof}
	For example, by calling up Definition . This works even if the ordering of things move.
	Note that the end of proof square box is already there
\end{proof}

\begin{theorem}
	And a final theorem
\end{theorem}

\begin{proof}
	Combining Definition  with Lemman we get Equation  below.
\end{proof}

\section{Including maths}

Some maths, like $\varepsilon>0$ or $a_{23}=\alpha^3$, is written in-line. More important or complex maths is displayed on its own line.
For example, $$ \lim_{x\to\infty}f(x)=\frac{\pi}{4}.$$

Sometimes you need multiple lines of maths to line up nicely:

\begin{align*}
f(x+y)&=(x+y,-2(x+y))\\
&=(x,-2x)+(y,-2y)\\
&=f(x)+f(y),
\end{align*}

and sometimes you want to number lines in an equation

\begin{align}
A^{T} & =\begin{pmatrix}1 & 2\\
3 & 4
\end{pmatrix}^{T}\\
\label{my_eqn}  & =\begin{pmatrix}1 & 3\\
2 & 4
\end{pmatrix}
\end{align}
