\chapter{Rumour Spreading on Static Networks}\label{chapter:staticIntro}

In this chapter, we introduce rumour spreading on static networks.

\section{Introduction to Rumour Spreading}

In this section we introduce the principles and terminology of rumour spreading common to all the rumour spreading processes we study in this report. Then we specify a rumour spreading algorithm for static networks.

Static rumour spreading takes place on a graph $G=(V,E)$ which we will often refer to as the "topology". The process takes place over time $t \geq 0$. Each node can be in one of two states at each time during the spread of a rumour, either it is aware of the rumour, or not aware of the rumour. For a given time $t$, we refer the set of nodes aware of the rumour as the "informed" nodes, denoted by $I_t$. The set of nodes not aware of the rumour at time $t$ are called the "uninformed" nodes, denoted by $U_t$. We assume that once a node is aware of the rumour, it never forgets it, so after a node been informed, it never becomes uniformed again. The rumour spreads from informed nodes to uninformed nodes along edges (sometimes referred to as "connections") between informed and uniformed nodes. We refer to such edges as "active" edges, as they have the potential to spread the rumour. The remaining edges between pairs of informed nodes or pairs of uninformed nodes are referred to as "inactive" edges, as the rumour cannot spread along these edges. 

We now introduce our first example of a rumour spreading algorithm, which specifies exactly how the rumour spreads.  

\begin{definition}\label{algo:staticAsync}
	Asynchronous Push-Pull Rumour Spreading Algorithm for Static Topologies

	\noindent
	This rumour spreading algorithm takes place in continuous time on a topology $G$ with $n$ nodes.
	Initially a single node of $G$ is aware of rumour. Each node is associated with an independent unit rate Poisson process. Say an arrival occurs at time $t$ in the Poisson process associated with node $v$. At this time if $v$ will contact one of its neighbours $u$ uniformly at random. If $v$ is aware of the rumour and $u$ is not, $v$ will push the rumour to $u$, and $u$ will become aware of the rumour immediately. Conversely, if $v$ is not aware of the rumour but $u$ is, $v$ will pull the rumour from $u$. If both nodes are aware of the rumour or both nodes are not aware of the rumour, nothing happens.
\end{definition}

% TODO: Mention WHAT IF DISCONNECTED - rumour never reach all nodes, speed at which rummour  spreading time complex to analyse since spreadds depends on current active edges (changes over time), depeneds on specific tradjectory. Poisson process makes analysis easier (see next chapter)

Since informed nodes never forget the rumour and the topology is connected, it is clear that eventually all nodes will be made aware of the rumour. We refer to the first time that all the nodes are aware of the rumour as the "spreading time". This is the central quantity we will bound in this report. The interpretation of the spreading time depends on the model. For example, in an epidemic model, this is the first time the entire population has been infected with the epidemic. Hence, the ability to predict and bound the spreading time has practical importance for decision-making in many scenarios where rumour spreading models are used.

\section{Bounding the spreading time}\label{section:graphMetrics}

In this section, we review a theorem for bounding the rumour spreading time of Algorithm \ref{algo:staticAsync}.

The bound we review is expressed in terms of the following functions of the topology, the values of which represent the strength of different structural properties of the given topology.

% TODO: Interpretation of the following metrics

\begin{definition}
	Conductance of a graph $G = (V, E)$
	$$
		\Phi(G) = \min_{\emptyset \neq S \subset V} \frac{|E(S, \comp{S})|}{\min\{\text{vol}(S), \text{vol}(\comp{S})\}}
	$$
\end{definition}

\begin{definition}
	Diligence of a cut $ E(S, \comp{S}) $
	$$
		\rho(S) = \comp{d}(S) \min_{\{u, v\} \in E(S, \comp{S}) } \left\{ \max \left\{ \frac{1}{d_u},\frac{1}{d_v} \right\} \right\}
	$$ 
	where $\comp{d}(S) := \frac{\sum_{v \in S} d_v}{|S|}$ is the average degree of the vertices in $S$
\end{definition}

\begin{definition}
	Diligence of a graph $G$
	$$
		\rho(G) = \min_{0 < \text{vol}(S) \leq \frac{\text{vol}(V)}{2}} \rho(S) 
	$$
\end{definition}

Now, we present a bound on the spreading time for a static network.
\begin{theorem}
	Let T be the spreading time for Algorithm \ref{algo:staticAsync} on a graph $G$ with $n$ vertices. Then for $c > 0$,
	$$
		\mathbb{P}\lb T \geq \frac{2c \log \frac{n}{2}}{\Phi(G)\rho(G)}\rb \leq \frac{1}{c}
	$$
\end{theorem}
Note that since that Algorithm \ref{algo:staticAsync} is randomised, the spreading time is a random variable, thus our bound is probabilistic. 
This theorem suggests that spreading time grows like 
$$
	\bigO \lb \frac{\log n}{\Phi(G)\rho(G)} \rb
$$ 
in the number of nodes of the topology.

% TODO: Generally interested in scaling in the number of vertices, r.v since ideally want bounds w.h.p (define), reserve n for number of nodes, prove by adapting 

% TODO: Example application to complete graph

% TODO: linking text

\section{Introducing Dynamic Networks}

Here we formally define the Dynamic Network structure rumours will spread on.

\begin{definition}
	Dynamic Network

	\noindent
	A dynamic network is a sequence of graphs $\mathcal{G} = (G_t)_{t \geq 0}$ indexed by an integer time $t$. All the graphs in the sequence have the same vertex set at each time step, but the edge set may vary, i.e.  $G_t = (V, E_t)$ where $E_t$ is some edge set on $V$.
\end{definition}

$G_t$ represents the topology of the network at the discrete time step $t$. However, asynchronous rumour spreading algorithms operate in continuous time, so we need to define the topology of the network at non-integer times. To represent the state of the network at any non-negative continuous time $\gamma \in \mathbb{R}_+$, we say that the current network topology $G_\gamma$ := $G_{\floor\gamma}$. Thus, for any time $\gamma \in [t, t + 1)$ the network topology is fixed to $G_t$. This corresponds to allowing the network topology change at integer time steps only.

% TODO: Segway

\begin{definition}
	Vertex degree at time $\gamma \in \mathbb{R}_+ $ 

	\noindent
	For a Dynamic Network $\mathcal{G}$ on a vertex set $V$, $d_v(\gamma)$ is the degree of a vertex $v \in V$ at time $\gamma$, i.e. the degree of $v$ in $G_\gamma$
\end{definition}

\begin{definition}
	Dynamic Network Isomorphism

	We say that two dynamic networks $\mathcal{G}_1 = (G_t)_{t \geq 0}$ and $\mathcal{G}_2 = (H_t)_{t \geq 0}$ are isomorphic if the topologies $G_t$ and $H_t$ are isomorphic for all $t \geq 0$. 
\end{definition}

% TOD: definition of active and inactive edges in static rumour spreading section.

% TODO: Discuss implications for async rumour spreading - interpreting poission process as exponential clock, not interested in value of the process

