\chapter{Rumour Spreading on Static Networks}\label{chapter:staticIntro}

In this chapter, we introduce rumour spreading on static networks.

\section{Introduction to Rumour Spreading}

In this section we introduce the principles and terminology of rumour spreading common to all the rumour spreading processes we study in this report. Then we specify a rumour spreading algorithm for static networks.

Static rumour spreading takes place on a graph $G=(V,E)$ which we will often refer to as the `topology'. The process takes place over time $t \geq 0$. Each node can be in one of two states at each time during the spread of a rumour, either it is aware of the rumour, or not aware of the rumour. For a given time $t$, we refer the set of nodes aware of the rumour as the `informed' nodes, denoted by $I_t$. The set of nodes not aware of the rumour at time $t$ are called the `uninformed' nodes, denoted by $U_t$. We assume that once a node is aware of the rumour, it never forgets it, so after a node been informed, it never becomes uniformed again. The rumour spreads from informed nodes to uninformed nodes along edges (sometimes referred to as `connections') between informed and uniformed nodes. We refer to such edges as `active' edges, as they have the potential to spread the rumour. The remaining edges between pairs of informed nodes or pairs of uninformed nodes are referred to as `inactive' edges, as the rumour cannot spread along these edges. We refer to the time at which a new node becomes aware of the rumour as a `spreading event'.

We now introduce our first example of a rumour spreading algorithm, which specifies exactly how the rumour spreads.  

\begin{definition}\label{algo:staticAsync}
	Asynchronous Push-Pull Rumour Spreading Algorithm for Static Topologies

	\noindent
	This rumour spreading algorithm takes place in continuous time on a topology $G$ with $n$ nodes.
	Initially a single node of $G$ is aware of rumour. Each node is associated with an independent unit rate Poisson process. Say an arrival occurs at time $t$ in the Poisson process associated with node $v$. At this time if $v$ will contact one of its neighbours $u$ uniformly at random. If $v$ is aware of the rumour and $u$ is not, $v$ will push the rumour to $u$, and $u$ will become aware of the rumour immediately. Conversely, if $v$ is not aware of the rumour but $u$ is, $v$ will pull the rumour from $u$. If both nodes are aware of the rumour or both nodes are not aware of the rumour, nothing happens.
\end{definition}

We consider the implications of this model. 
Note that if the network on which the rumour spreads is disconnected, by Algorithm \ref{algo:staticAsync} there will always be some nodes which are unaware of the rumour. However, suppose the network is connected.
Since informed nodes never forget the rumour, it is clear that eventually all nodes will be made aware of the rumour. We refer to the first time that all the nodes are aware of the rumour as the `spreading time'. This is the central quantity we will bound in this report. The interpretation of the spreading time depends on the model. For example, in an epidemic model, this is the first time the entire population has been infected with the epidemic. Hence, the ability to predict and bound the spreading time has practical importance for decision-making in many scenarios where rumour spreading models are used.

However, we can already observe challenges in bounding the spreading time. The speed at which the rumour spreads at a given time $t$ is dependent on the set of active edges. However, this set itself is random and dependent on the state of the network at time $t$. The number of potential states the network could be in grows exponentially with the number of nodes, since each node can be in one of two states so for a network with $n$ nodes there are $2^n$ potential states. The state of the network is also dependent on the history of the rumour spread.
Since the process is asynchronous, there are an uncountably infinite number of ways for the rumour to spread. All of these features make the spreading time process difficult to analyse. However, in Chapter \ref{chapter:AsyncUpperBound}, we will see techniques to tackle this complexity. We also note that the choice of the Poisson process as the fundamental stochastic process driving the rumour spread is essential for simplifying the analysis.

\section{Bounding the spreading time}\label{section:graphMetrics}

In this section, we review a theorem for bounding the rumour spreading time of Algorithm \ref{algo:staticAsync}.

The bound we review is expressed in terms of the following functions of the topology, the values of which represent the strength of different structural properties of the given topology. We collectively refer to these functions as `connectivity metrics'.

\begin{definition}
	Conductance of a graph $G = (V, E)$
	$$
		\Phi(G) = \min_{\emptyset \neq S \subset V} \frac{|E(S, \comp{S})|}{\min\{\text{vol}(S), \text{vol}(\comp{S})\}}
	$$
\end{definition}

This function measures how `well-connected' a graph is. We observe that the conductance of disconnected graphs is 0, since if we take one of the connected components to be $S$ the cut set $E(S, \comp{S})$ will be empty. Connected graphs have a positive conductance up to a maximum of $\frac{1}{2}$, which is achieved by the complete graph where all nodes are connected to each other (we compute this conductance in Theorem \ref{theorem:ringCompleteAsyncBound}). In Chapter \ref{chapter:AsyncUpperBound}, we see how the conductance affects the spreading time. 

\begin{definition}
	Diligence of a cut $ E(S, \comp{S}) $
	$$
		\rho(S) = \comp{d}(S) \min_{\{u, v\} \in E(S, \comp{S}) } \left\{ \max \left\{ \frac{1}{d_u},\frac{1}{d_v} \right\} \right\}
	$$ 
	where $\comp{d}(S) := \frac{\sum_{v \in S} d_v}{|S|}$ is the average degree of the vertices in $S$
\end{definition}

\begin{definition}
	Diligence of a graph $G$
	$$
		\rho(G) = \min_{0 < \text{vol}(S) \leq \frac{\text{vol}(V)}{2}} \rho(S) 
	$$
\end{definition}

The diligence measures the variation in the degree of the nodes in a cut. The diligence of any connected graph where all the nodes have the same degree is 1. The diligence takes a maximum value of 2, by considering the graph where a single node is connected by $n$ edges to $n$ other nodes. In this case, the cut which achieves the minimum separates the single high degree node and the $n$ low degree nodes. In Chapters \ref{chapter:AsyncUpperBound} and \ref{chapter:asyncBoundTight}, we see how this quantity affects the spreading time. 

Now, we present a bound on the spreading time for a static network using these connectivity metrics.

\begin{theorem}\label{theorem:staticBound}
	Let T be the spreading time for Algorithm \ref{algo:staticAsync} on a graph $G$ with $n$ vertices. Then for $c > 0$,
	$$
		\mathbb{P}\lb T \geq \frac{2c \log \frac{n}{2}}{\Phi(G)\rho(G)}\rb \leq \frac{1}{c}
	$$
\end{theorem}

We do not prove this theorem as it is a simple consequence of a standard result (see \cite{complexNetworksRumourSpreading}) adapted to include the diligence. We will see many of the key ideas behind this theorem in Chapter \ref{chapter:AsyncUpperBound}.

We give an example of applying Theorem \ref{theorem:staticBound} bound to a static network.

\begin{theorem}
	Let T be the spreading time for Algorithm \ref{algo:staticAsync} on the complete graph $K_n$ with $n$ vertices. Then for a $c > 0$,
	$$
		\mathbb{P}\lb T \geq 4c \log \frac{n}{2} \rb \leq \frac{1}{c} 
	$$
\end{theorem}

\begin{proof}
	In Theorem \ref{theorem:ringCompleteAsyncBound}, we will see that $\Phi(K_n) = \frac{1}{2}$ and $\rho(K_n) = 1$. Substituting these values yields the result.
\end{proof}

Note that since that Algorithm \ref{algo:staticAsync} is randomised, the spreading time is a random variable, thus our bound is probabilistic. 
For a fixed number of nodes $n$, by taking $c$ arbitrarily large, we get an arbitrarily confident bound.
For the remainder of this report, the variable $n$ denotes the number of nodes in the given network.
This theorem suggests that spreading time on the complete graph grows like 
$
	\bigO \lb \log n \rb
$
in the number of nodes $n$ of the topology. In this report, we are interested in the spreading time of large networks, when alternatives such as simulation become infeasible. Thus, we are only concerned with how the spreading time scales in $n$, as for large $n$, constants become irrelevant when comparing asymptotic growth.

For the other bounds in this report, we demand a stronger result.

\begin{definition}
	With high probability (w.h.p)

	\noindent
	Let $\lb A_n \rb_{n \geq 0}$ be a sequence of events indexed by an integer $n$. We say the events occur with high probability (w.h.p) if for some $c \geq 1$, 
	$$
		\mathbb{P}(A_n) \geq 1 - n^{-c}
	$$ for all sufficiently large $n$.
\end{definition}

This is a strong requirement since for large $n$, the probability that bound holds will be infinitesimally close to 1. However, we can be extremely confident in bounds that hold w.h.p, as for large $n$, it is overwhelming likely that they hold.
