\chapter{Example of an Adversarial Network}\label{chapter:asyncBoundTight}

In Section \ref{section:asyncEvaluation}, simulations suggested that the asynchronous bound is almost tight in some adversarial cases. In this chapter, we go a step further, and present a family of adversarial networks for which the upper bound derived in Chapter \ref{chapter:AsyncUpperBound} is provably almost tight. This construction was given in \cite{asyncPaper}, however we add detail to help the reader understand the analysis. First we introduce the family of networks. Then we evaluate the spreading time bound for this family. Finally, we derive a lower bound on the spreading time and show that the bound is indeed almost tight.

\section{Introducing the Adversarial Network}

Before we define the adversarial network itself, we introduce a family of topologies used to construct the network.

\begin{definition}\label{def:HkAB}
 	Graph $H_{k, \Delta}(A,B)$

	Let $V$ be a set of $n$ vertices, with $A \subset V$ an arbitrary subset satisfying $\floor*{\frac{n}{4}} \leq |A| \leq \ceil*{\frac{3n}{4}}$.
	Let B denote $V \setminus A$. The positive integers $k = \mathcal{O}\left(\frac{\log n}{\log \log n}\right)$ and $\Delta = \mathcal{O}(\sqrt{n})$ are fixed. We construct the graph $H_{k, \Delta}(A,B)$ on the vertex set $V$ using the following steps

	\textit{Step 1.} Let $\{S_i, 0 \leq i \leq k\}$ denote an arbitrary set of disjoint subsets of nodes in $V$ such that $|S_i| = \Delta$ for all $i$, $S_0 \subseteq A$, and $S_i \subseteq B$ for $1 \leq i \leq k$. We connect each node of $S_i$ to all nodes in $S_{i+1}$ for $0 \leq i \leq k - 1$. This generates a string of $k$ complete bipartite graphs, as seen in Figure \ref{fig:HkAB_1}.

    \begin{figure}[h]
        \centering
        \begin{tikzpicture}[
    roundnode/.style={circle, fill=black, minimum size=0.5mm},
    align=center,
    node distance=0.5cm and 2cm,
    line width = 0.3mm
    ]
    %Nodes
    \node[roundnode] (1A) {};
    \node[roundnode] (1B) [below=of 1A] {};
    \node[roundnode] (1C) [below=of 1B] {};
    \node[roundnode] (1D) [below=of 1C] {};
    
    \node[draw=blue!20,inner sep=2mm,label=above:$S_0$,fit=(1A) (1A) (1A) (1D)] (S0fit) {};

    \node[draw=red!20,inner sep=6mm,label=below:$A$,fit=(S0fit)] {};
    
    \node[roundnode] (2A) [right=of 1A]{};
    \node[roundnode] (2B) [below=of 2A] {};
    \node[roundnode] (2C) [below=of 2B] {};
    \node[roundnode] (2D) [below=of 2C] {};
    
    \node[draw=blue!20,inner sep=2mm,label=above:$S_1$,fit=(2A) (2A) (2A) (2D)] (S1fit) {};
    
    \node[roundnode] (3A) [right=of 2A]{};
    \node[roundnode] (3B) [below=of 3A] {};
    \node[roundnode] (3C) [below=of 3B] {};
    \node[roundnode] (3D) [below=of 3C] {};
    
    \node[draw=blue!20,inner sep=2mm,label=above:$S_2$,fit=(3A) (3A) (3A) (3D)] {};
    
    \node[roundnode] (4A) [right=of 3A]{};
    \node[roundnode] (4B) [below=of 4A] {};
    \node[roundnode] (4C) [below=of 4B] {};
    \node[roundnode] (4D) [below=of 4C] {};
    
    \path (3A) -- node[auto=false]{\ldots} (4A);
    \path (3B) -- node[auto=false]{\ldots} (4B);
    \path (3C) -- node[auto=false]{\ldots} (4C);
    \path (3D) -- node[auto=false]{\ldots} (4D);
    
    \node[draw=blue!20,inner sep=2mm,label=above:$S_{k-1}$,fit=(4A) (4A) (4A) (4D)] {};
    
    \node[roundnode] (5A) [right=of 4A]{};
    \node[roundnode] (5B) [below=of 5A] {};
    \node[roundnode] (5C) [below=of 5B] {};
    \node[roundnode] (5D) [below=of 5C] {};
    
    \node[draw=blue!20,inner sep=2mm,label=above:$S_k$,fit=(5A) (5A) (5A) (5D)] (Skfit) {};
    
    \node[draw=red!20,inner sep=6mm,label=below:$B$,fit=(S1fit) (Skfit)] {};

    %Lines
    \draw[-] (1A) -- (2A);
    \draw[-] (1B) -- (2A);
    \draw[-] (1C) -- (2A);
    \draw[-] (1D) -- (2A);
    \draw[-] (1A) -- (2B);
    \draw[-] (1B) -- (2B);
    \draw[-] (1C) -- (2B);
    \draw[-] (1D) -- (2B);
    \draw[-] (1A) -- (2C);
    \draw[-] (1B) -- (2C);
    \draw[-] (1C) -- (2C);
    \draw[-] (1D) -- (2C);
    \draw[-] (1A) -- (2D);
    \draw[-] (1B) -- (2D);
    \draw[-] (1C) -- (2D);
    \draw[-] (1D) -- (2D);
    
    
    \draw[-] (2A) -- (3A);
    \draw[-] (2B) -- (3A);
    \draw[-] (2C) -- (3A);
    \draw[-] (2D) -- (3A);
    \draw[-] (2A) -- (3B);
    \draw[-] (2B) -- (3B);
    \draw[-] (2C) -- (3B);
    \draw[-] (2D) -- (3B);
    \draw[-] (2A) -- (3C);
    \draw[-] (2B) -- (3C);
    \draw[-] (2C) -- (3C);
    \draw[-] (2D) -- (3C);
    \draw[-] (2A) -- (3D);
    \draw[-] (2B) -- (3D);
    \draw[-] (2C) -- (3D);
    \draw[-] (2D) -- (3D);
    
    \draw[-] (4A) -- (5A);
    \draw[-] (4B) -- (5A);
    \draw[-] (4C) -- (5A);
    \draw[-] (4D) -- (5A);
    \draw[-] (4A) -- (5B);
    \draw[-] (4B) -- (5B);
    \draw[-] (4C) -- (5B);
    \draw[-] (4D) -- (5B);
    \draw[-] (4A) -- (5C);
    \draw[-] (4B) -- (5C);
    \draw[-] (4C) -- (5C);
    \draw[-] (4D) -- (5C);
    \draw[-] (4A) -- (5D);
    \draw[-] (4B) -- (5D);
    \draw[-] (4C) -- (5D);
    \draw[-] (4D) -- (5D);

    \end{tikzpicture}
        \caption{Construction of $H_{k, \Delta}(A,B)$ after Step 1}
        \label{fig:HkAB_1}
    \end{figure} 

	\textit{Step 2.} Let $G_A$ be an arbitrary 
	graph on $A \setminus S_0$ such that $\Phi(G_A) = \Theta(1)$ and all nodes have a $\Theta(1)$ neighbours. % MAYBE: Cosntruction??
	We now connect each node of $S_0$ to $\Delta$ distinct nodes of $G_A$. First, we arbitrarily enumerate the nodes in $S_0$ and $G_A$ as $v_1, v_2, \dots, v_\Delta$ and $u_1, u_2, \dots, u_{|G_A|}$ respectively. Connect $v_1$ to the first $\Delta$ nodes in the $G_A$, starting with $u_1$. Connect $v_2$ to the next $\Delta$ nodes in $G_A$, starting with $u_{\Delta + 1}$.
	Repeat this process for each $v_i$, as in Figure \ref{fig:HkAB_2}.
	If there are more new connections than nodes in $G_A$, reuse nodes in $G_A$ starting again from $u_1$ and forming new connections in the enumerated order, i.e, the $i^\text{th}$ new connection should be made with node $u_m$ where $m = i\mod |G_A|$.

    \begin{figure}[h]
        \centering
        \begin{tikzpicture}[
    roundnode/.style={circle, fill=black, minimum size=0.5mm},
    align=center,
    node distance=0.5cm and 2cm,
    redline/.style={-, red},
    blueline/.style={-, blue},
    greenline/.style={-, green}
    ]
    %Nodes
    \node[roundnode, label=west:$u_1$] (1A) {};
    \node[roundnode, label=west:$u_2$] (1B) [below=of 1A] {};
    \node[roundnode, label=west:$u_\Delta$] (1C) [below=of 1B] {};
    \node[roundnode, label=west:$u_{\Delta+1}$] (1D) [below=of 1C] {};
    \node[roundnode] (1E) [below=of 1D] {};

    \node[inner sep=3mm,label=above:$G_A$,fit=(1A) (1A) (1A) (1E)] {};
    
    \node[roundnode, label=east:$v_1$, red] (2A) [right=of 1A]{};
    \node[roundnode, label=east:$v_2$, blue] (2B) [below=of 2A] {};
    \node[roundnode, label=east:$v_\Delta$, green] (2C) [below=of 2B] {};
    
    \node[inner sep=3mm,label=above:$S_0$,fit=(2A) (2A) (2A) (2C)] {};
    
    %Lines
    \draw[redline] (1A) -- (2A);
    \draw[redline] (1B) -- (2A);
    \draw[redline] (1C) -- (2A);

    \draw[blueline] (1D) -- (2B);
    \draw[blueline] (1E) -- (2B);
    \draw[blueline] (1A) -- (2B);

    \draw[greenline] (1B) -- (2C);
    \draw[greenline] (1C) -- (2C);
    \draw[greenline] (1D) -- (2C);

\end{tikzpicture}
        \caption{Illustration of the enumeration technique used in Step 2}
        \label{fig:HkAB_2}
    \end{figure}

	\textit{Step 3.} Let $G_B$ be another arbitrary graph on $B \setminus \bigcup_{i=1}^k S_i$ such that $\Phi(G_B) = \Theta(1)$ and each node has $\Theta(1)$ neighbours. 
	We connect each node of $S_k$ to $\Delta$ distinct nodes of $G_B$ using the same enumeration method as in Step 2. This yields the graph $H_{k, \Delta}(A,B)$ illustrated in Figure \ref{fig:HkAB_3}.

    \begin{figure}[h]
        \centering
        \begin{tikzpicture}[
    roundnode/.style={circle, fill=black, minimum size=0.5mm},
    align=center,
    node distance=0.5cm and 2cm,
    line width = 0.3mm
    ]
    %Nodes
    \node[roundnode] (1A) {};
    \node[roundnode] (1B) [below=of 1A] {};
    \node[roundnode] (1C) [below=of 1B] {};
    \node[roundnode] (1D) [below=of 1C] {};
    \node[roundnode] (1E) [below=of 1D] {};

    \node[draw=blue!20,inner sep=2mm,label=above:$G_A$,fit=(1A) (1A) (1A) (1E)] {};
    
    \node[roundnode] (2A) [right=of 1A]{};
    \node[roundnode] (2B) [below=of 2A] {};
    \node[roundnode] (2C) [below=of 2B] {};
    \node[roundnode] (2D) [below=of 2C] {};
    
    \node[draw=blue!20,inner sep=2mm,label=above:$S_0$,fit=(2A) (2A) (2A) (2D)] {};
    
    \node[roundnode] (3A) [right=of 2A]{};
    \node[roundnode] (3B) [below=of 3A] {};
    \node[roundnode] (3C) [below=of 3B] {};
    \node[roundnode] (3D) [below=of 3C] {};
    
    \node[draw=blue!20,inner sep=2mm,label=above:$S_1$,fit=(3A) (3A) (3A) (3D)] {};
    
    \node[roundnode] (4A) [right=of 3A]{};
    \node[roundnode] (4B) [below=of 4A] {};
    \node[roundnode] (4C) [below=of 4B] {};
    \node[roundnode] (4D) [below=of 4C] {};
    
    \path (3A) -- node[auto=false]{\ldots} (4A);
    \path (3B) -- node[auto=false]{\ldots} (4B);
    \path (3C) -- node[auto=false]{\ldots} (4C);
    \path (3D) -- node[auto=false]{\ldots} (4D);
    
    \node[draw=blue!20,inner sep=2mm,label=above:$S_k$,fit=(4A) (4A) (4A) (4D)] {};
    
    \node[roundnode] (5A) [right=of 4A]{};
    \node[roundnode] (5B) [below=of 5A] {};
    \node[roundnode] (5C) [below=of 5B] {};
    \node[roundnode] (5D) [below=of 5C] {};
    \node[roundnode] (5E) [below=of 5D] {};
    \node[roundnode] (5F) [below=of 5E] {};
    
    \node[draw=blue!20,inner sep=2mm,label=above:$G_B$,fit=(5A) (5A) (5A) (5F)] {};
    
    
    %Lines
    \draw[-] (1A) -- (2A);
    \draw[-] (1B) -- (2A);
    \draw[-] (1C) -- (2A);
    \draw[-] (1D) -- (2A);

    \draw[-] (1A) -- (2B);
    \draw[-] (1B) -- (2B);
    \draw[-] (1C) -- (2B);
    \draw[-] (1E) -- (2B);

    \draw[-] (1A) -- (2C);
    \draw[-] (1B) -- (2C);
    \draw[-] (1D) -- (2C);
    \draw[-] (1E) -- (2C);

    \draw[-] (1A) -- (2D);
    \draw[-] (1C) -- (2D);
    \draw[-] (1D) -- (2D);
    \draw[-] (1E) -- (2D);
    
    \draw[-] (2A) -- (3A);
    \draw[-] (2B) -- (3A);
    \draw[-] (2C) -- (3A);
    \draw[-] (2D) -- (3A);
    \draw[-] (2A) -- (3B);
    \draw[-] (2B) -- (3B);
    \draw[-] (2C) -- (3B);
    \draw[-] (2D) -- (3B);
    \draw[-] (2A) -- (3C);
    \draw[-] (2B) -- (3C);
    \draw[-] (2C) -- (3C);
    \draw[-] (2D) -- (3C);
    \draw[-] (2A) -- (3D);
    \draw[-] (2B) -- (3D);
    \draw[-] (2C) -- (3D);
    \draw[-] (2D) -- (3D);
    
    \draw[-] (4A) -- (5A);
    \draw[-] (4A) -- (5B);
    \draw[-] (4A) -- (5C);
    \draw[-] (4A) -- (5D);

    \draw[-] (4B) -- (5E);
    \draw[-] (4B) -- (5F);
    \draw[-] (4B) -- (5A);
    \draw[-] (4B) -- (5B);

    \draw[-] (4C) -- (5C);
    \draw[-] (4C) -- (5D);
    \draw[-] (4C) -- (5E);
    \draw[-] (4C) -- (5F);

    \draw[-] (4D) -- (5A);
    \draw[-] (4D) -- (5B);
    \draw[-] (4D) -- (5C);
    \draw[-] (4D) -- (5D);

    \end{tikzpicture}
        \caption{Construction of $H_{k, \Delta}(A,B)$ after Step 3}
        \label{fig:HkAB_3}
    \end{figure}

\end{definition}

We note that Definition \ref{def:HkAB} technically specifies a family of graphs, since the construction relies on choosing arbitrary sets of vertices, each of which yields a distinct graph. However, for the purposes of this analysis we think of $H_{k, \Delta}(A,B)$ as a single graph as all graphs in the family have the same structural properties required for the proof, so we can choose any of them.

Using $H_{k, \Delta}(A,B)$, we now construct the following family of networks for which Theorem \ref{theorem:AsyncUpperBound} is tight.

\begin{definition}
	$\rho$-diligent Dynamic Network
	
	First we define the sequences of node subsets $(A_t)_{t \geq 0}$ and $(B_t)_{t \geq 0}$ with $A_t, B_t \subseteq V$ for all $t$. For each $t$, we construct $A_t$ and $B_t$ such that they partition $V$ into two subsets, as follows:

	Let $A_0$ be an arbitrary subset of $V$ of size $\floor*{\frac{n}{4}}$ containing the first node to be aware of the rumour. 
	Set $B_0 = V \setminus A_0$. Let $B_{t+1} = B_t \setminus I_{t+1}$, i.e. the set of nodes in $B_t$ that were still not informed of the rumour by round $t+1$.
	Let $A_{t+1} = V \setminus A_t$, i.e. the set of nodes that were either in $A_0$ or informed of the rumour by round $t+1$.

	Now we can define the dynamic network $\rho$-diligent Dynamic Network $\mathcal{G}(n, \rho) = (G_t)_{t \geq 0}$ itself. 

	Let $\Delta = \ceil*{\frac{1}{\rho}}$, and $k = \mathcal{O}(\log n / \log \log n)$.
	Let $G_t = H_{k, \Delta}(A_t, B_t)$ while $|B_t| \geq \frac{n}{4}$. Note that since $\Delta = \mathcal{O}(\sqrt{n})$ by the definition of $H_{k, \Delta}(A,B)$, we must have that $\rho = \Omega\left(\frac{1}{\sqrt{n}}\right)$.
	Note also that $|A_t|$ is increasing in $t$ since $A_t$ is a superset of the informed nodes at time $t$, so there may be some time step $l$ where $|A_l| > \ceil*{\frac{3n}{4}}$. Thus, for all time steps $t \geq l$, we cannot construct $H_{k, \Delta}(A_t,B_t)$. Hence, for all $t \geq l$, let $G_{t+1} = G_t$.
\end{definition}

Note that this network reacts to the spread of the rumour. At the end of each round, when the topology is allowed to change, all the informed nodes are moved into the set $A$. In Section \ref{section:adversarialLowerBound} we will see how this adversarial behaviour prolongs the rumour spreading time.

\section{Obtaining the Upper Bound}

In this section, we use Theorem \ref{theorem:AsyncUpperBound} to bound the rumour spreading time of Algorithm \ref{NodeCentricAsyncAlgorithm} on a $\rho$-diligent dynamic network. 

Note that since the network is allowed to react to the rumour, and the rumour spreading process is random, the sequence of topologies is randomised. However, we will see that since all topologies have the same connectivity metrics regardless of the rumour spread, we can still apply Theorem \ref{theorem:AsyncUpperBound} for the $rho$-diligence network.

\begin{theorem}
	The rumour spreading time of Algorithm \ref{NodeCentricAsyncAlgorithm} on a $\rho$-diligent dynamic network with $n$ vertices, where $\rho = \Omega(\frac{1}{\sqrt{n}})$, is at most 
	$$
		\mathcal{O}\left(\left(\rho n + \frac{k}{\rho}\right)\log n\right)
	$$
	w.h.p
\end{theorem}

\begin{proof}
	To determine a concrete bound using Theorem \ref{theorem:AsyncUpperBound}, we need to calculate $\Phi(G_t)$ and $\rho(G_t)$. Since $G_t = H_{k, \Delta}(A, B)$ for some $A$ and $B$ at each time step, it suffices to calculate the conductance and diligence for a general $H_{k, \Delta}(A, B)$.

	\textbf{Claim 1.} $\Phi(H_{k, \Delta}(A, B)) = \Theta\left(\frac{\Delta^2}{k\Delta^2 +n }\right)$

	For $q = 1, \dots, k$, let $A_q = G_A \cup \bigcup_{i=1}^q  S_i$. We want to calculate $\text{vol}(A_q)$. The degree of any arbitrary node in $S_i$ is $2\Delta$, since it is connected to all $\Delta$ nodes in $S_{i-1}$ and all $\Delta$ nodes in $S_{i+1}$, where we set $S_0 = G_A$ and $S_{k+1} = G_B$ for brevity. Since there are $\Delta$ nodes in each $S_i$, $\text{vol}(S_i) = 2\Delta^2$. By taking the disjoint union of $q$ $S_i$ subsets, we obtain that $\text{vol}(\bigcup_{i=1}^q) = 2q\Delta^2$. Now we consider $\text{vol}(G_A)$. Since $|A| = \Theta(n)$ and $|S_i| = \Delta = \mathcal{O}(\sqrt{n})$, 
	$$
		|G_A| = |A \setminus S_0| = |A| - |S_0| = \Theta(n) - \mathcal{O}(\sqrt{n}) = \Theta(n)
	$$
	By construction each node in $G_A$ is connected to $\Theta(1)$
	other nodes in $G_A$, so internal edges contribute $\Theta(n)$ degree to $\text{vol}(G_A)$. The only other edges which terminate in $G_A$ are the $\Delta^2$ edges between $G_A$ and $S_0$, thus $\text{vol}(G_A) = \Delta^2 + \Theta(n)$. By taking the disjoint union of $G_A$ and $\bigcup_{i=1}^q S_i$ we obtain $\text{vol}(A_q) = (2q + 1)\Delta^2 + \Theta(n)$. 
	
	Note that for any $q$, $|E(A_q,\comp{A_q})| = \Delta^2$ since the only edges connecting $A_q$ and $\comp{A_q}$ are the edges passing between $S_q$ and $S_{q+1}$. This is most easily seen by inspecting Figure \ref{fig:HkAB_3}, and noting that $A_q$ corresponds to the first $q + 1$ sets of nodes in the string, while $\comp{A_q}$ corresponds to the remaining $k - q + 1$ sets of nodes at the end of the string.
	Hence, we can derive an upper bound on the conductance of $H_{k, \Delta}(A, B)$ by evaluating the conductance expression for the set $A_q$ as follows.
	$$
		\Phi(H_{k, \Delta}(A, B)) 
		\leq \frac{|E(A_q, \comp{A_q})|}{\min \left\{ \text{vol}(A_q), \text{vol}(\comp{A_q}) \right\} } 
		= \frac{\Delta^2}{\min \left\{ \text{vol}(A_q), \text{vol}(\comp{A_q}) \right\}}
	$$	
	To get as tight a bound as possible, we choose $q$ to maximise the denominator. Note that by symmetry, $\vol(\comp{A_q}) = 2(k - q + 1)\Delta^2 + \Theta(n)$. Thus, the denominator is maximised by taking $q = \floor*{\frac{k}{2}}$. Hence,
	$$
		\Phi(H_{k, \Delta}(A, B)) 
		\leq \frac{\Delta^2}{2(k - \floor*{\frac{k}{2}} + 1)\Delta^2 + \Theta(n)}
		= \Theta\lb\frac{\Delta^2}{k\Delta^2 + n}\rb
	$$
	In fact, $A_\frac{k}{2}$ is a set of nodes which minimises the conductance. Intuitively this fact seems reasonable. $G_A$ and $G_B$ are well-connected sets by construction, so when choosing $S$ to minimise $|E(S, \comp{S})|$, it is desirable to avoid making a cut through either of these sets. Additionally, choosing $A_\frac{k}{2}$ separates the total volume of the graph in half, which maximises $\min \{\vol(S),\vol(\comp{S})\}$, and minimises the overall expression. 
	
	However, proving that no set achieves a lower conductance is a challenge. Let us ignore $G_A$ and $G_B$ for now and try and compute the conductance of the string of $k+1$ bipartite graphs $B$. Let $S$ be an arbitrary set of nodes and let $R_i = S_i \cap S$. We compute a general formula for the conductance of the cut $S$ as follows.
	\begin{align*}	
		\frac{|E(S, \comp{S})|}{\min\{\vol(S), \vol(\comp{S})\}} &=
		\frac{\sum_{i=0}^{k-1} |E(\comp{R_i}, R_{i+1})| + |E(R_i, \comp{R_{i+1}})|}{\min\{\vol(S), \vol(\comp{S})\}} \\
		\intertext{by considering each bipartite segment}
		&=
		\frac{\sum_{i=0}^{k-1} |\comp{R_i}||R_{i+1}| + |R_i||\comp{R_{i+1}}|}{\min\{\vol(S), \vol(\comp{S})\}}
		\intertext{by the structure of the network (refer to Figure \ref{fig:HkAB_3})}
		&=
		\frac{\sum_{i=0}^{k-1} |\comp{R_i}||R_{i+1}| + |R_i||\comp{R_{i+1}}|}{2\Delta\min\{\sum_{i=0}^{k-1} |R_i|, \sum_{i=0}^{k-1} |\comp{R_i}|\}}
		\intertext{since each node has degree $2\Delta$}
		&= \frac{\sum_{i=0}^{k-1} (\Delta - |R_i|)|R_{i+1}| + |R_i| (\Delta - |R_{i+1}|)}{2\Delta\min\{\sum_{i=0}^{k-1} |R_i|, \sum_{i=0}^{k-1} (\Delta - |R_i|)\}} \\
		\intertext{since $|S_i|=\Delta$ for all $i \leq k$}
	\end{align*}
	This expression shows that due to the symmetry of the network, the conductance only depends on how many nodes of $S$ are contained in each $S_i$. However, it is difficult to optimise and bound, demonstrating why the lower bound on the conductance is hard to prove for the full topology. Hence, we omit this half of the proof.

	% MAYBE: MAke k = \Theta(log n / log log n )
	\textbf{Claim 2.} $\rho(H_{k, \Delta}(A,B)) = \Theta\left(\frac{1}{\Delta}\right)$

	First we consider the set of vertices $A = G_A \cup S_0$ for which the minimal diligence is achieved.
	We have that $\text{vol}(A) = 2\Delta^2 + \Theta(n)$ since $S_0$ contributes $2\Delta^2$ degree and $G_A$ contributes $\Theta(n)$ degree as shown in claim 1. Since $\Delta = \mathcal{O}(\sqrt{n})$, we obtain that 
	$$
		\text{vol}(A) = 2 \mathcal{O}(\sqrt{n})^2 + \Theta(n) = \mathcal{O}(n) + \Theta(n) = \Theta(n)
	$$ 
	Since $|A| = \Theta(n)$ by construction we have that the average degree of $A$ satisfies $$
		\bar{d}(A) = \frac{\text{vol}(A)}{|A|} = \frac{\Theta(n)}{\Theta(n)} = \Theta(1)
	$$
	To compute the diligence of $A$, we now consider the cut set $E(A, \comp{A})$. Since every endpoint of the cut set has degree $2\Delta$, we get that 
	$$
		\min_{\{u, v\} \in E(A, \comp{A})} \max \left\{ \frac{1}{d_u}, \frac{1}{d_v} \right\} = \frac{1}{2\Delta}
	$$
	Hence, the diligence of $A$ is 
	$$
		\rho(A) = \frac{\bar{d}(A)}{2\Delta} = \Theta\left(\frac{1}{\Delta}\right)
	$$ 
	
	Now we prove that no set with a smaller diligence exists. First note that we increase the degree of each node in $G_A$ by $\Theta(1)$ when joining them to $S_0$ and $S_1$ in Step 2 of the construction of $H_{k, \Delta}(A,B)$. To see this, we observe that the enumeration technique illustrated in Figure \ref{fig:HkAB_2} spreads the $\Delta^2$ additional edges evenly across $G_A$. Thus, the degree increase for an arbitrary node in $G_A$ is at most
	$$
		\ceil*{\frac{\Delta^2}{|G_A|}}=\frac{\mathcal{O}(n)}{\Theta(n)} = \mathcal{O}(1)
	$$
	since $\Delta = \mathcal{O}(\sqrt{n})$ and $|G_A| = \Theta(n)$. 
	
	Since by construction, the internal edges in $G_A$ contribute $\Theta(1)$ degree to each node, we have that the degree for an arbitrary node in $G_A$ is $\Theta(1) + \mathcal{O}(1) = \Theta(1)$. As $G_B$ is connected to the string of bipartite graphs in an identical fashion during Step 2, all nodes of $G_B$ also have $\Theta(1)$ degree.  
	
	Let $S \subseteq V$ be an arbitrary subset of nodes. Since we have that all nodes in $G_A$ and $G_B$ have degree $\Theta(1)$, the maximum degree of any node in the graph is $2\Delta$ for large $n$. Hence 

	$$
	\min_{\{u, v\} \in E(S, \comp{S}) } \left\{ \max \left\{ \frac{1}{d_u},\frac{1}{d_v} \right\} \right\} \geq \frac{1}{2\Delta}
	$$
	Since all nodes have at least constant degree, we have that $\comp{d}(S) = \Omega(1)$. Thus, 
	$$
		\rho(S) 
		= \comp{d}(S) \min_{\{u, v\} \in E(S, \comp{S}) } \left\{ \max \left\{ \frac{1}{d_u},\frac{1}{d_v} \right\} \right\}
		\geq \frac{\Omega(1)}{2\Delta}
		= \Omega\left(\frac{1}{\Delta}\right)
	$$
	Thus, since no subset of nodes exists with a lower diligence than the set $A$, our claim holds.

	We can now apply the bound from Theorem \ref{theorem:AsyncUpperBound}, to yield an upper bound on the spreading time of a $\rho$-diligent dynamic network $\mathcal{G}(n, \rho) = (G_t)_{t\geq 0}$. 
	
	From Theorem \ref{theorem:AsyncUpperBound}, we have that the first $T$ for which $\sum_t^T \Phi(G_t)\rho(G_t)$ exceeds $C \log n = \mathcal{O}(\log n)$ is an upper bound on the spreading time w.h.p.  Note that 
	$$
		\sum_t^T \Phi(G_t)\rho(G_t)
		= 
		T \Theta\left(\frac{\Delta^2}{k\Delta^2 +n }\right) \Theta\left(\frac{1}{\Delta}\right)
		= 
		T \Theta\left(\frac{\Delta}{k\Delta^2 +n }\right)
		= 
		T \Theta\left(\frac{1}{n \rho + \frac{k}{\rho}}\right)
	$$
	since $G_t = H_{k, \Delta}(A_t,B_t)$ for all $t$, where $\Delta = \ceil*{\frac{1}{\rho}} = \Theta\left(\frac{1}{\rho}\right)$ by the definition of the dynamic network. Hence 
	$$
		\min \left\{T : \sum_t^T \Phi(G_t)\rho(G_t) \geq C \log n \right\}
		=
		\frac{\mathcal{O}(\log n)}{\Theta\left(\frac{1}{n \rho + \frac{k}{\rho}}\right)}
		= 
		\mathcal{O}\left(\log n \left(n \rho + \frac{k}{\rho}\right)\right)
	$$
\end{proof}

\section{Obtaining the lower bound}\label{section:adversarialLowerBound}

In this section we verify that the upper bound is tight by computing a lower bound on the spreading time for a $\rho$-diligent dynamic network that holds w.h.p.

First we prove a lemma about the spread of the rumour down the chain of bipartite subgraphs in $H_{k, \Delta}(A,B)$.

\begin{lemma}\label{lemma:H_k,DeltaABOneStep}
	Suppose we spread a rumour according to Algorithm \ref{NodeCentricAsyncAlgorithm} on $G = H_{k, \Delta}(A,B)$ for some $A, B$ with $\Delta = \ceil*{\frac{1}{\rho}}$. Assume that every node in $S_0$ is aware of the rumour, and no node in $B$ is informed. Then the probability that at least one node of $S_k$ is informed after a unit length of time is at most
	$$
		\frac{2^k}{k!}\ceil*{\frac{1}{\rho}}
	$$	
\end{lemma}

\begin{proof}
	We prove this lemma by considering a modified rumour spread process.
	Let $\{v_i, v_{i+1}\}$ be an arbitrary edge between $S_i$ and $S_{i+1}$. Since each of the nodes in $\cup_i^k S_i$ has $2\Delta$ neighbours, by superposition, $v_i$ and $v_{i+1}$ communicate according to a Poisson process with rate $\frac{1}{2\Delta} + \frac{1}{2\Delta} = \frac{1}{\Delta}$. We consider the following equivalent algorithm, which we call the 2-push algorithm. Associate each node with a Poisson process of rate 2. Each node pushes the rumour (if they know it) to a neighbour chosen uniformly at random according to the arrival times of the associated Poisson process. Under this algorithm, by thinning, we see that $v_i$ and $v_{i+1}$ communicate according to a Poisson process with rate $\frac{2}{2\Delta} = \frac{1}{\Delta}$, so the 2-push algorithm is equivalent to the original process.

	Now we introduce the modified algorithm we analyse in the remainder of the proof, known as the forward 2-push algorithm. In this algorithm, each node with a Poisson process of rate 2. If $u \in S_i$ is aware of the rumour, according to the arrival times of its associated Poisson process, it pushes the rumour to a randomly chosen neighbour in $S_{i+1}$. Under this algorithm, the rumour only moves in one direction down the chain of $S_i$s.

	Let $\varepsilon_1$, $\varepsilon_2$, $\varepsilon_3$ denote the probabilities that least one node of $S_k$ is informed after a unit length of time under Algorithm \ref{NodeCentricAsyncAlgorithm}, the 2-push algorithm, and the forward 2-push algorithm respectively.
	We want to show that  
	$$
		\mathbb{P}(\varepsilon_1) \leq \mathbb{P}(\varepsilon_3)
	$$
	so that we can analyse the simplified forward 2-push algorithm instead of Algorithm \ref{NodeCentricAsyncAlgorithm}.
	Since Algorithm \ref{NodeCentricAsyncAlgorithm} is equivalent to the 2-push algorithm, we have that $
	\mathbb{P}(\varepsilon_1) = \mathbb{P}(\varepsilon_2)$. Now we show that $
	\mathbb{P}(\varepsilon_2) \leq \mathbb{P}(\varepsilon_3)$ by stochastic domination and coupling.

	\textbf{Claim. } $\mathbb{P}(\varepsilon_2) \leq \mathbb{P}(\varepsilon_3)$

	\noindent
	Let $X$ denote the random variable of the largest index $m$ such that a node of $S_m$ is aware of the rumour under the 2-push algorithm. Let $Z$ denote the same random variable under the forward 2-push algorithm. We present a coupling $(\tilde{X}, \tilde{Z})$ of $X$ and $Z$. 
	Consider an arbitrary trajectory of the rumour under the 2-push algorithm. Here, we think of the trajectory as a sequence of directed active edges along which spreading events took place. In general, the trajectory of rumour will branch when the same node informs multiple nodes. We build a trajectory for the forward 2-push algorithm one branch at a time, where a branch runs from the initial node that was aware of the rumour, to the final node to be informed. Suppose we have the branch trajectory $e_1, e_2, \dots, e_l$ starting a $u_0 \in S_0$. Then at the times of the spreading events, we construct a branch trajectory for the 2-push algorithm by choosing the next node in the trajectory uniformly at random from the next layer of nodes, starting at $u_0$. Let $\tilde{X}$ denote the random variable of the largest index $m$ such that a node of $S_m$ is aware of the rumour under the original trajectories, and let $\tilde{Z}$ denote the same random variable under the modified trajectories. It is clear that $\tilde{X} \stackrel {d}{=} X$ since both random variables are generated by the same process. In the original trajectories, the process moves forwards according to a Poisson process of rate of $1$ and backwards according to a Poisson process of rate of $1$, by superposition. The modified trajectories move forward whenever the original trajectory moves forward or backwards, thus by superposition under these trajectories each node pushes the rumour forward according to a Poisson process of rate of $2$. Hence, $\tilde{Z} \stackrel{d}{=} Z$, and $(\tilde{X}, \tilde{Z})$ is indeed a coupling of $X$ and $Z$. 
	
	Suppose $\tilde{X} = m$. Then there must be a branch trajectory such that the final node is in $S_m$. Note that since the associated modified branch trajectory moved forward whenever the original trajectory moved forward or backwards, this trajectory must end in $S_l$ for some $l \geq m$. Hence, 
	$$
		\tilde{X} = m \leq l \leq \tilde{Z}
	$$
	By Theorem \ref{theorem:couplingDomination}, we have that $X \preceq Z$, so
	$$
		\mathbb{P}(\varepsilon_1) 
		= \mathbb{P}(X \geq k) 
		\leq \mathbb{P}=(Z \geq k) \mathbb{P}(\varepsilon_3)
	$$
	Thus, the claim is proved.

	Now we analyse the forward 2-push algorithm.

	Let $I_i(\gamma)$ denote the number of informed nodes in $S_i$ at time $\gamma \in [0,1]$. 

	We prove the lemma by induction on $\mathbb{N}$.

	\underline{Base Case.} $\mathbb{E}I_1(\tau) \leq 2 \tau \Delta$

	Since each node in $S_0$ is aware of the rumour, each of the $\Delta^2$ edges from $S_0$ to $S_1$ is pushing the rumour forward at rate $\frac{2}{\Delta}$ for all times $\gamma \in [0,1]$. Thus, by the superposition property of Poisson processes, the overall rate at which the rumour is pushed from $S_0$ to any node of $S_1$ is $\frac{2}{\Delta}\Delta^2 = 2\Delta$. Let $N_1(\tau)$ denote the number of times any node of $S_1$ is informed of the rumour in the interval $[0,\tau]$ (even if it already knew the rumour). By the theory of Poisson processes, $N_1(\tau)$ is distributed according to a Poisson distribution with mean $2\tau\Delta$. Note $I_i(\tau) \leq N_1(\tau)$ since the same nodes of $S_1$ may be contacted multiple times. Hence, $\mathbb{E}I_1(\tau) \leq \mathbb{E}N_1(\tau) = 2\tau\Delta$ as required.
	
	\underline{Inductive Case.} $\mathbb{E}I_k(\tau) \leq \frac{2^k \tau^k}{k!}\Delta$

	Assume that for all $m \leq k - 1$ and $\gamma \in [0,1]$, $\mathbb{E}I_m(\gamma) \leq \frac{2^m \gamma^m}{m!}\Delta$ holds.
	By the tower law,
	$$
		\mathbb{E}I_k(\tau) = \mathbb{E}\mathbb{E}[I_k(\tau) |I_{k-1}(\gamma), \gamma \in [0, \tau]]
	$$
	As in the base case, let $N_k(\tau)$ denote the number of times a node in $S_k$ is informed of the rumour in the interval $[0, \tau]$. Since $I_k(\tau) \leq N_k(\tau)$,
	$$
		\mathbb{E}[I_k(\tau) |I_{k-1}(\gamma), \gamma \in [0, \tau]] \leq \mathbb{E}[N_k(\tau) |I_{k-1}(\gamma), \gamma \in [0, \tau]]
	$$
	At time $\gamma \in [0,1]$, there are $I_{k-1}(\gamma)$ nodes aware of the rumour in $S_{k-1}$, and pushing the rumour to $S_k$ at rate 2. By the superposition of Poisson processes, the nodes of $S_{k-1}$ inform the nodes of $S_k$ according to a Poisson process of rate $2 I_{k-1}(\gamma)$.
	Thus, $N_k(\tau)|(I_{k-1}(\gamma), \gamma \in [0, \tau])$ is an inhomogeneous Poisson process, where the value of the rate function at time $\gamma$ is $2 I_{k-1}(\gamma)$. By the definition of the inhomogeneous Poisson process,
	\begin{align*}
		\mathbb{E}\mathbb{E}[N_k(\tau)|I_{k-1}(\gamma), \gamma \in [0, \tau]] &= \mathbb{E}\int_0^\tau 2 I_{k-1}(\gamma) d\gamma \\
		&= 2 \int_0^\tau \mathbb{E}I_{k-1}(\gamma) d\gamma \\ 
		\intertext{where we can swap the integral and expectation by Fubini's Theorem (see \cite{fubini}) since $I_{k-1}(\gamma) \leq \Delta$ and $\tau \leq 1$}
		& \leq 2 \int_0^\tau \frac{2^{k-1} \gamma^{k-1}}{(k-1)!} \Delta d\gamma \\
		\intertext{by the induction hypothesis}
		& = \frac{2^k \tau^k}{k!} \Delta
	\end{align*}
	Hence,
	$$
		\mathbb{E}I_k(\tau) \leq \mathbb{E}\frac{2^k \tau^k}{k!}\Delta = \frac{2^k \tau^k}{k!}\Delta
	$$
	Let $k=2ec\frac{\log n}{\log \log n}$ and $\tau = 1$, since we are interested in the spread by the end of the unit time round. By taking logarithms and applying Stirling's Approximation for the $\log (k!)$ term, we have that $\mathbb{E}I_k(1) \leq n^{-c}$.
	See \cite{stirling} for a full proof of Stirling's Approximation. 
	Thus, by Markov's inequality,
	$$
		\mathbb{P}(\varepsilon_1) \leq \mathbb{P}(\varepsilon_3) = \mathbb{P}(I_k(1) \geq 1) \leq \mathbb{E}I_k(1) \leq \frac{2^k}{k!}\Delta
	$$
\end{proof}

\begin{theorem}
	w.h.p, the rumour spreading time of Algorithm \ref{NodeCentricAsyncAlgorithm} on a $\rho$-diligent dynamic network with $n$ vertices, where $\rho = \Omega(\frac{1}{\sqrt{n}})$ and $k = \Theta\lb\frac{\log n}{\log \log n}\rb$ is at least 
	$$
		\Omega\left(\frac{n \rho}{k}\right)
	$$
\end{theorem}

% MAYBE: Interesting dependence on diligence -> higher diligence = longer to spread?

\begin{proof} 
	Let $T_0$ be the first time that all the nodes of $A_{T_0}$ are aware of the rumour. To simplify our analysis, suppose that we start at time $T_0$, hence for all remaining time steps we can assume that every node in $S_0$ is aware of the rumour.

	By the definition of a $\rho$-diligent network, at time $t=0$  we have that all the informed nodes are located in $A_0$, and no node of $B_0$ knows the rumour. 
	
	Thus, by applying Lemma \ref{lemma:H_k,DeltaABOneStep}, with probability $n^{-c}$ at time $t = 1$, no node of $S_k$ is informed of the rumour, for some constant $c \geq 2$. We omit the evaluation of Lemma \ref{lemma:H_k,DeltaABOneStep} for our value of $k$ for brevity, but note that the approach is to take logarithms and apply Stirling's Formula \cite{stirling}. Since the only way the rumour can reach the node set $G_B = B \setminus \bigcup_{i=1}^k S_i$ is through $S_k$ (see Figure \ref{fig:HkAB_3}), we can conclude that w.h.p after the first time step no node of $G_B$ is aware of the rumour. Equivalently we can say that the set of nodes $B_0 \cap I_1 \subseteq \bigcup_i^k S_i$, i.e. all nodes of $B_0$ that are aware of the rumour after one time step are confined to the set of $S_i$s in $G_1$. Hence, 
	\begin{equation}\label{eq:firstStepinfectedNodesOfB}
		|B_0 \cap I_1| \leq \left|\bigcup_{i=1}^k S_i\right| = k\Delta
	\end{equation}
	since each of the $k$ disjoint node sets $S_i$ have size $\Delta$. Since $B_1 = B_0 \setminus I_1$
	$$
		|B_1| = |B_0 \setminus I_1| = |B_0| - |B_0 \cap I_1| \geq \frac{3n}{4} - k\Delta
	$$
	where the final inequality holds due to inequality \ref{eq:firstStepinfectedNodesOfB} and the fact that $|B_0| \geq \frac{3n}{4}$. 

	We now repeat this analysis for the first $\frac{n}{4k\Delta}$ time steps, by first showing that w.h.p no node in $S_k$ is made aware of the rumour in the first $\frac{n}{4k\Delta}$ time steps. Let $\mathcal{A}_t$ denote the event that in time step $t$ some node in $S_k$ is made aware of the rumour. Note that the event 
	$$\mathcal{A} = \left\{S_k \text{ is informed of the rumour any time step } t \leq \frac{n}{4k\Delta} \right\} = \bigcup_{t=1}^\frac{n}{4k\Delta} \mathcal{A}_t
	$$
	hence by the union bound and Lemma \ref{lemma:H_k,DeltaABOneStep},
	$$
		\mathbb{P}(\mathcal{A}) 
		\leq \sum_{t=1}^\frac{n}{4k\Delta} \mathbb{P}(A_t) 
		\leq \frac{n}{4k\Delta}n^{-c}
		= \frac{n^{-(c-1)}}{4k\Delta} \leq n^{-(c-1)}
	$$
	as $k, \Delta \geq 1$. Since $c \geq 2$
	we have that w.h.p in the first $\frac{n}{4k\Delta}$ time steps the rumour does not reach $S_k$.

	Thus, for $m \leq \frac{n}{4k\Delta}$, $|B_{m-1} \cap I_m| \leq k\Delta$ so by induction it follows that,
	$$
		|B_m| = |B_{m-1} \setminus I_m| \geq \frac{3n}{4} - (m-1)k\Delta - k\Delta = \frac{3n}{4} - mk\Delta
	$$
	Note that for $m = \frac{n}{4k\Delta}$, we get that 
	$$
		|B_m| \geq \frac{3n}{4} - \frac{nk\Delta}{4k\Delta} = \frac{n}{2}
	$$
	Thus, w.h.p after $\frac{n}{4k\Delta}$ time steps, there are still at least $\frac{n}{2}$ nodes who are not aware of the rumour (since no node in $B_i$ is aware of the rumour at the beginning of each time step by construction). Hence, we can conclude the algorithm takes at least 
	$$
		\Omega\left(\frac{n}{4k\Delta}\right) = \Omega\left(\frac{n\rho}{k}\right)
	$$
	time to spread the rumour to all nodes of $\mathcal{G}(n, \rho)$, where $\Delta = \ceil*{\frac{1}{\rho}}$.
\end{proof}

\begin{corollary}
	For all large $n$ and arbitrary $\rho = \Omega\left(\frac{1}{\sqrt{n}}\right)$, there exists a dynamic network on $n$ vertices with diligence $\rho$ for every topology, for which the bound is at most an $o\lb \log^2 n \rb$ factor larger than the true spreading time w.h.p.
\end{corollary}

\begin{proof}
	Recall that by Theorem \ref{theorem:AsyncUpperBound}, the spreading time is at most
	$$
		\mathcal{O}\left(\log n \left(n \rho + \frac{k}{\rho}\right)\right)
	$$
	w.h.p.
	Since $\rho \in \Omega\left(\frac{1}{\sqrt{n}}\right)$ and  $k \in \bigO \lb \frac{\log n}{\log \log n} \rb$, we have that
	\begin{align*}
		\frac{\bigO \lb \log n \lb n \rho + \frac{k}{\rho} \rb \rb}{\Omega \lb \frac{n\rho}{k} \rb} 
		&= \bigO \lb \frac{\log n \lb n \rho + \frac{k}{\rho} \rb}{\frac{n\rho}{k}} \rb \\
		&= \bigO \lb k \log n + \frac{k^2 \log n}{n \rho^2} \rb \\
		&= \bigO \lb \frac{(\log n)^2}{\log \log n}  + \frac{(\log n)^3}{(n \log \log n)^2} \rb \\
		&= o\lb \log^2 n \rb
	\end{align*}
	
\end{proof}

Since the bound is larger than the true spreading time by at most a poly-logarithmic factor, we say the bound is almost tight for the $\rho$-diligent network family. 

Overall, in this chapter we have shown that the asynchronous bound is useful in some cases. Through the extended example, we have become much more familiar with adversarial behaviours in dynamic networks, and saw that the static network is not always the adversarial isomorphism of a given network.