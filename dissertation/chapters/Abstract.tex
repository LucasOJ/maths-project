\begin{abstract}
    
Rumour spreading is an important stochastic process used to model many real-world processes such as the spread of epidemics and file sharing on computer networks [CITE]. % TODO: USED process twice
Rumour spreading on static networks is well understood [CITE]. However, in many 
situations the topologies on which the rumour spreads change over time [CITE]. This motivates the study of rumour spreading on dynamic networks.
The dual aims of this report are to prove and evaluate probabilistic upper bounds on the rumour spreading time of dynamic networks.
We introduce 2 models for rumour spreading on dynamic networks, one which operates in continuous time and another which operates in discrete time. We denote these models the asynchronous and synchronous models, respectively. For each model, we evaluate the associated upper bound by comparing it with simulation results for a selection of networks. We claim that both bounds are useful by analysing extended examples of networks where the bounds are provably tight. % TODO: SOMETHING ABOUT USE
However, simulations reveal that the asynchronous bound is weak for many networks. We also find that the synchronous bound is only applicable to networks satisfying strong structural conditions, thus has limited use.

% TODO: Why study spreading time.
% TODO: Evaulating what about the models.

\end{abstract}