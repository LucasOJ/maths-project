\begin{abstract}
    
Rumour spreading is an important stochastic process used to model many real-world systems, such as the spread of epidemics and file sharing on computer networks \cite{motivation}.
Rumour spreading on static networks is well understood \cite{staticsWellUnderstood}. However, in many 
situations the topologies on which the rumour spreads change over time \cite{motivation}. This motivates the study of rumour spreading on dynamic networks.
The dual aims of this report are to prove and evaluate probabilistic upper bounds on the rumour spreading time of dynamic networks. 
We introduce two models for rumour spreading on dynamic networks. The first, introduced by Pourmiri and Mans in \cite{asyncPaper}, operates in continuous time. The second, introduced by Clementi et al. in \cite{syncPaper}, operates in discrete time. We denote these models the asynchronous and synchronous models respectively. For each model, we evaluate the associated upper bound by comparing it with simulation results for a selection of networks. After analysing extended examples of networks where the bounds are provably tight, we claim that both bounds are useful in some cases.
However, simulations reveal that the asynchronous bound is weak for many networks. We also find that the synchronous bound is only applicable to networks satisfying strong structural conditions, thus both bounds have limitations.

% MAYBE: Why study spreading time.

\end{abstract}