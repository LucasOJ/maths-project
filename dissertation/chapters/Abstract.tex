\begin{abstract}
    
Rumour spreading is an important stochastic process used to model many real-world processes such as the spread of epidemics and file sharing on computer networks [CITE]. Rumour spreading on static networks is well understood [CITE]. However, in many 
scenarios the topologies on which the rumours spread change over time [CITE]. This motivates the study of rumour spreading on dynamic networks.
The aims of this report are to prove and evaluate probabilistic upper bounds on the rumour spreading time of dynamic networks.
We introduce 2 models for rumour spreading on dynamic networks, one which operates in continuous time and another which operates in discrete time, known as the asynchronous and synchronous models respectively. For each model we evaluate the associated upper bound by comparing it with simulation results for a selection of networks. We also see extended examples of networks where the bounds are provably tight, and therefore useful. However, simulations reveal that the asynchronous bound is weak for many networks. We also find that the synchronous bound is only appliable to networks satisfying strong structural conditions, and therefore as limited use.

% TODO: Why study spreading time.

\end{abstract}