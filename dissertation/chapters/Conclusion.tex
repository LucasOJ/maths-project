\chapter{Conclusion}\label{chapter:conclusion}

In this section, we summarise the report, highlight the central results, and present ideas for further study.

We started by introducing an asynchronous rumour spreading model for static networks in Chapter \ref{chapter:staticIntro}, and presented a well-known bound on the spreading time. 

Then, in Chapter \ref{chapter:AsyncUpperBound}, we introduced dynamic networks, and presented a randomised algorithm for rumour spreading on such networks. In Theorem \ref{theorem:AsyncUpperBound}, we proved that the rumour spreading time can be bounded w.h.p. using connectivity metrics of the network's topologies, namely the conductance and diligence. We saw that this bound has no use for networks consisting solely of disconnected topologies. An interesting question for further study is whether it is possible to construct spreading time bounds for networks consisting of disconnected topologies. By applying the bound to various networks and comparing it with simulated spreading times, we saw that the tightness of the bound varies depending on the network. For the networks under investigation, we found that the bound was almost tight in adversarial cases, but much weaker in isomorphic non-adversarial networks. 

In Chapter \ref{chapter:asyncBoundTight}, we strengthened the conclusions of Chapter \ref{chapter:AsyncUpperBound} by constructing a family of adversarial networks for which the bound is provably almost tight (Corollary \ref{corollary:edgeMarkovianThetaBound}). We saw that these networks take advantage of dynamic capabilities to slow the spread of the rumour.

In Chapter \ref{chapter:SyncFlooding} we considered an alternative, deterministic model for rumour spreading known as flooding. In Theorem \ref{theorem:DeterministicFloodingBound}, we presented a bound on the spreading time which also depends on the connectivity metrics of the network. 
However, this bound requires strong connectivity properties to hold for every topology in the network.

We then introduced Markovian-Evolving Dynamic Networks (MEDNs), and generalised the flooding bound to this class of randomised networks in Theorem \ref{theorem:markovSyncBound}. We applied the generalised bound to a specific MEDN known as the Edge-Markovian network, with the restriction that $\hat{p} \geq \frac{c\log n}{n}$ for a sufficiently large $c$. Constructing upper bounds on the spreading time for the Edge-Markovian network $\hat{p} \leq \frac{c \log n}{n}$ remains an interesting area for further study. In Corollary \ref{corollary:edgeMarkovianThetaBound}, we showed that the bound was tight by proving a lower bound on the spreading time that holds w.h.p. We concluded the chapter by investigating how the bound behaved for Edge-Markovian networks with different parameters. Here we saw that for the largest value of $\hat{p}$ where the bound is provably tight, the bound did not match the average spreading times as well as it did for simulations with smaller values of $\hat{p}$. Why this is the case forms another interesting question for further study.

% TODO: SWEEP for future study in readthrough