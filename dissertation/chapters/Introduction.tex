\chapter{Introduction}

Dynamic networks are complex networks which change over time. The study of such structures is important as we can use them to model many important real-world situations \cite{motivation}. For example, computer networks such as the internet are dynamic: connections between nodes can fail and new nodes can be added which leads to a changing network topology. 

In this paper we investigate two rumour spreading algorithms on dynamic networks. Both of these algorithms begin with a single node aware of the rumour, which spreads between nodes along present edges. These algorithms model important processes such as epidemics spreading through a population \cite{staticsWellUnderstood}, and distributed databases replicating their state across a network \cite{stateReplication}. 

We compare rumour spreading algorithms by deriving the time it takes for the rumour to spread to all nodes on a given dynamic network, known as the rumour spreading time. 
For simplicity, we restrict our investigations to dynamic networks where the edges may be introduced or removed over time, but the vertex set remains the same. 

In Chapter \ref{chapter:Prelims} we introduce the mathematical preliminaries needed to model and analyse rumour spreading on dynamic networks.

In Chapter \ref{chapter:staticIntro} we formally introduce the concept of rumour spreading, and review a standard result which bounds the rumour spreading time on static networks.

In Chapter \ref{chapter:AsyncUpperBound} we start by introducing the Dynamic Network model. Then, we specify an asynchronous (i.e. continuous time) rumour spreading algorithm for this model, and review a result by Pourmiri and Mans \cite{asyncPaper} which bounds the associated spreading time. In this analysis the exact changes in network topology are known before the rumour spreading takes place, however in practice this assumption may be unrealistic. We also evaluate the quality of the bound by applying the bound to a selection of networks and comparing the results with spreading times from simulations.

In Chapter \ref{chapter:asyncBoundTight} we review the complementary result from the same paper \cite{asyncPaper} that the asynchronous bound is almost tight for a family of dynamic networks.

In Chapter \ref{chapter:SyncFlooding} we investigate the flooding time for a synchronous (i.e. discrete time) rumour spreading algorithm, following the results by of a paper by Clementi et al. \cite{syncPaper}. We loosen the assumption that we know the exact changes to the network topology at each time step, by instead considering a model where changes occur according to a given probability distribution. We conclude with an extended example of applying the synchronous bound to a random network, and a comparison with simulated spreading times. 

In Chapter \ref{chapter:conclusion}, we summarise the key results of this report and draw final conclusions.