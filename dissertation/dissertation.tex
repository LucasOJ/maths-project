\documentclass[a4paper,11pt]{article}

\usepackage{amsmath}
\usepackage{amssymb}
\usepackage{amsthm}
\usepackage{graphicx}
\usepackage{enumerate}
\usepackage{mathtools}

%you can add more packages using the same code above

%------------------

%\setlength{\topmargin}{0.0in}
%\setlength{\textheight}{10in}
%\setlength{\oddsidemargin}{0.0in}
%\setlength{\evensidemargin}{0.0in}
%\setlength{\textwidth}{6.5in}

%-------------------


\newtheorem{theorem}{Theorem}[section]
\newtheorem{proposition}[theorem]{Proposition}
\newtheorem{lemma}[theorem]{Lemma}
\newtheorem{corollary}[theorem]{Corollary}
\newtheorem{conjecture}[theorem]{Conjecture}


\theoremstyle{definition}
\newtheorem{definition}[theorem]{Definition}
\newtheorem*{example}{Example}

\newcommand*\comp[1]{\overline{#1}}

\newcommand{\lb}{\left}
\newcommand{\rb}{\right}


\DeclarePairedDelimiter\ceil{\lceil}{\rceil}
\DeclarePairedDelimiter\floor{\lfloor}{\rfloor}

%------------------

%Everything before begin document is called the pre-amble and sets out how the document will look
%It is recommended you don't touch the pre-amble until you are familiar with LateX

\begin{document}

\begin{titlepage}
\thispagestyle{empty}

\begin{figure}[h]
\begin{center}
\includegraphics[scale=0.5]{graphics/uob_logo.pdf} %make sure the pdf file named 'uob' is saved in the same folder as this file
\end{center}
\end{figure}

\begin{center}
{\Large Bounding the Rumour-Spreading Time on Dynamic Networks\\ \vspace{1cm}Lucas O'Dowd-Jones}
\end{center}

\vspace{3cm}
\hrule
\begin{center}
Supervised by Ayalvadi Ganesh\\
Level M\\
40 Credit Points
\end{center}
\hrule

\vspace{3cm}
\begin{center}
\today
\end{center}

\end{titlepage}



\begin{abstract}
	TODO
\end{abstract}
	
	
\tableofcontents


\section{Introduction}

Dynamic networks are complex networks which change over time. The study of such structures are important as we can use them to model many important real-world situations[CITE]. For example, computer networks such as the internet are dynamic: connections between nodes can fail and new nodes can be added which leads to an ever-changing network topology. 

In this paper we investigate a variety of randomized rumour spreading algorithms on dynamic networks. In these algorithms initially a single node is aware the rumour, which spreads randomly between nodes along present edges. These algorithms model important phenomenons such as epidemics spreading through a population [CITE], or distributed databases replicating their state across a network [CITE]. 

% TODO: FOLLOWING IS WRONG - ALSO CONSIDERING FLOODING TIME
We compare rumour spreading algorithms by deriving the time it takes for the rumour to spread to all nodes on a given dynamic network, known as the rumour spreading time. Since the algorithms we consider are randomized, the rumour spreading time will vary. Hence, the best we can do is we find bounds that hold with high probability. Such bounds are well-known for static networks [CITE], but in these paper we extend these results to networks changing topologies. For simplicity, we restrict our investigations to dynamic networks where the edges may be added or deleted, but the vertex set remains the same.

In Section \ref{AsyncUpperBoundSection} we review a result by [CITE] which bounds the spreading time of an asynchronous rumour spreading algorithm. In this analysis the exact changes in network topology are known before the rumour spreading takes place. If we do not control the changes this assumption may be unrealistic.

In Section \ref{AsyncLowerBoundSection} we review the complementary result from the same paper that the asynchronous bound is almost tight. 

In Section \ref{SyncFloodingSection} we investigate the flooding time for a synchronous rumour spreading algorithm, following results of a paper by [CITE]. We also loosen the assumption that we know the exact changes to the network topology at each time step. Instead, we consider a model where changes occur according to a given probability distribution.

In Section \ref{AsyncCheegerBound} we combine ideas from sections \ref{AsyncUpperBoundSection} and \ref{SyncFloodingSection} to derive a novel bound on the rumour spreading time of an asynchronous algorithm. 

% TODO - finish this

\begin{enumerate}
	\item Motivation for studying
	\item Structure of paper
	\item Results proved
	\item My contributions
\end{enumerate}

\section{Bounding the asynchronous rumour spreading time in terms of the conductance}
\label{AsyncUpperBoundSection}

\subsection{Review of Possion Processes}

TODO

\begin{enumerate}
	\item Definition
	\item $N(s + t) - N(s)$ has a Poisson distribution with rate $\lambda t$
	\item Superposition
\end{enumerate}

\subsection{Introduction to Inhomogeneous Poisson Processes}

\begin{enumerate}
	\item Definition
	\item Proof that $N(b) - N(a)$ has a Poisson distribution with rate
	$$
		\int_a^b \lambda(t) dt
	$$
\end{enumerate}

\subsection{Order statistics of exponential random variables}

TODO: Prove that minimum of exponentials is exponential, where the probability each exponential that attains the minimum is dependent on each exponentials' rate, and independent of the value of the min


\newcommand{\ModelIntro}{
	Let $\mathcal{G} = (G_t)_{t \in \mathbb{N}}$ be an $n$-node dynamic network, where one node is aware of a rumour in $G_0$.
}

\subsection{Model}

\begin{definition}
	Dynamic Network

	$$
		\mathcal{G} = (G_t)_{t \in \mathbb{N}}, G_t = (V, E_t)
	$$
\end{definition}


In this subsection, we define the model used for asynchronous rumour spreading


An asynchronous rumour spreads on a dynamic network $\mathcal{G} = (G_t)_{t\in \mathbb{N}}$ in rounds. In round $t$, each node aware of the rumour is associated with a unit rate exponential clock. When the clock \dots

TODO: Finish specification of the model

\subsection{Proof of the bound}

\begin{definition}
	Cut
	$$
		E(S, \comp{S}) = \left\{\{u, v\} \in E \mid u \in S, v \in \comp{S} \right\}
	$$
\end{definition}

\begin{definition}
	Diligence of a cut $ E(S, \comp{S}) $
	$$
		\rho(S) = \comp{d}(S) \min_{\{u, v\} \in E(S, \comp{S}) } \left\{ \max \left\{ \frac{1}{d_u},\frac{1}{d_v} \right\} \right\}
	$$ 

	where $\comp{d}(S) := \frac{\sum_{v \in S} d_v}{|S|}$ is the average degree of the vertices in $S$
\end{definition}

\begin{definition}
	Diligence of a graph $G$
	$$
		\rho(G) = \min_{S \in V} \rho(S) 
	$$
\end{definition}

\begin{definition}
	Absolute Diligence of a non-empty graph $G = (V, E)$
	$$
		\comp{\rho}(G) = \min_{\{u, v\}v \in E}\left\{ \max \left\{ \frac{1}{d_u},\frac{1}{d_v} \right\} \right\}
	$$
\end{definition}

\begin{definition}
	Conductance of a graph $G = (V, E)$

	$$
		\Phi(G) = \min_{\emptyset \neq S \subset V} \frac{|E(S, \comp{S})|}{\min\{\text{vol}(S), \text{vol}(\comp{S})\}}
	$$
\end{definition}


First we prove a lemma about how long it takes for the number of nodes aware of the rumour to increase by a multiplicative factor. 

\begin{lemma} \label{AsyncIncreaseLemma}
	Let $\mathcal{G}=(G_t)_{t \in \mathbb{N}}$ be an $n$-node dynamic network, where some node is aware of the rumour in $G_0$. Choose an arbitrary time $\tau \in [0, \infty)$. We denote the set of informed nodes at this time $I_\tau$, and set uninformed nodes $U_\tau$.
	\noindent
	Define 
	
	$$
	\Delta(\alpha) = \min \left\{t: \sum_{k=0}^t \Phi(G_{\ceil{\tau}+k})\rho(G_{\ceil{\tau}+k}) \geq 2 \alpha \right\}
	$$

	\noindent
	Let $\tau'$ denote the earliest time when $I_t$ increases by $m(\tau) := \frac{\min\{|I_\tau|, |U_\tau|\}}{2}$.
	\noindent
	Then, 
	$$
		\mathbb{P}(\tau' - \tau \geq \Delta(\alpha) + 2) \leq e^{-c\alpha m(\tau)}
	$$
	\noindent
	where $c = \frac{1}{2} - \frac{1}{e}$
\end{lemma}

\begin{proof}
	Let $\gamma \in [\tau', \tau)$. We consider the set of edges $E(I_\gamma, U_\gamma)$ between the informed and uninformed sets of nodes at time $\gamma$. For every $\{u, v\} \in E(I_\gamma, U_\gamma)$, node $u$ pushes the rumour to $v$ according to a Poisson process of rate $\frac{1}{d_v(\gamma)}$
	
	
	% Mention later:
	Since the topology of the network only changes at integer times, we have that $G_\gamma = G_{\lfloor\gamma\rfloor}$. 
\end{proof}

Now we iteratively apply Lemma \ref{AsyncIncreaseLemma} to derive the following Theorem.

\begin{theorem}
	\ModelIntro Then, with high probability, the rumour will have spread to all the nodes of $\mathcal{G}$ in time at most

	$$
		\min \left\{t : \sum_{k=0}^t \Phi(G_k)\rho(G_k) \geq C \log n \right\} 
	$$

	\noindent
	for a sufficiently large $C$
\end{theorem}

TODO: Proof

\begin{proof}

\end{proof}

\begin{theorem}
	\ModelIntro Then with high probability, the rumour will have spread to all the nodes of $\mathcal{G}$ in time at most 

	$$ 
		\min \left\{ t : \sum_{k=0}^t \delta_{G_k} \comp{\rho}(G_k) \right\}
	$$

	%TODO: FINISH THEORM STATEMENT EG
	% for a sufficiently large $C$
\end{theorem}

TODO: Proof 

\begin{proof}
	
\end{proof}

\subsection{Interpretation}

TODO

\begin{enumerate}
	\item Why diligence useful?
	\item When useful/better than other bounds?
	\item Comparison between absolute and non-absolute  diligence results, when is each useful?
\end{enumerate}

\subsection{Simulations}

TODO 

\begin{enumerate}
	\item Look at symmetrical graphs (complete, star, ring, path) and evaluate explicit bounds - evaluate tightness in each case with simulations
	\item Simulations for random graphs/evolutions
\end{enumerate}

\section{Proof that the asynchronous rumour spreading time bound is almost tight}
\label{AsyncLowerBoundSection}

\section{Bounding the synchronous flooding time}
\label{SyncFloodingSection}

\subsection{Review of Markov Chains}

\begin{enumerate}
	\item Definition
	\item Stationary distribution definition and Interpretation
\end{enumerate}

\subsection{Model}

TODO: Describe how synchronous rumour spreads on dynamic network

\subsection{Proof of the bound}

First we give a bound on the rumour spreading time for a deterministic sequence of graphs. 

\begin{theorem}
	\ModelIntro Suppose also that there exists an increasing sequence $1 = h_0 \leq h_1 < \dots < h_s = \frac{n}{2}$ and decreasing sequence $k_1 \geq k_2 \geq \dots \geq k_s$ of positive real numbers, such that for all $t \in \mathbb{N}$, $G_t$ is a $(h_i, k_i)$-expander for every $i \in \{1, \dots , s\}$. Then the flooding time of $\mathcal{G}$ is

	$$
		\mathcal{O}\left(\sum_{i=1}^s \frac{\log \frac{h_i}{h_{i-1}}}{\log(1+k_i)}\right)
	$$
\end{theorem}

TODO: Finish proof

\begin{proof}
	For $i = 1,\dots, s$, let $T_i$ be the earliest time step such that the number of informed nodes is larger than $h_i$,

	$$
		T_i = \min \{ t \in \mathbb{N} : |I_t| > h_i \}
	$$

	Suppose $t$ is a time step for which $h_{i-1} < |I_t| \leq h_i$ for some $i = 1,\dots, s$. Since $G_t$ is a $(h_i, k_i)$-expander and $|I_t| \leq h_i$, the number of nodes at the next time step is 
	
	\begin{align*}
		|I_{t+1}| &= |I_t| + N(I_t) \\
		& \geq |I_t| + k_i |I_t| \\
		& = (1 + k_i)|I_t|
	\end{align*}

	% TODO: expand proof for this
	Hence, by induction, after $s$ time steps


	% TODO: Why does |I_{t+s}| satisfy required condtition to apply inequality?
	% \begin{align*}
	% 	|I_{t+s}| &\geq (1 + k_i)|I_{t+s-1}| \\
	% 	&\geq (1 + k_i)^2|I_{t+s-2}|\\
	% 	& \dots \\
	% 	&\geq (1 + k_i)^s |I_t| \\
	% 	&\geq (1 + k_i)^s h_{i-1}
	% \end{align*}

\end{proof}

\subsection{Applying the bound to Edge-Markovian evolution}

TODO: Prove that stationary distribution of edge-markovian evolution is ER G(n, p/(p+q))

\begin{theorem}
	Let $\mathcal{M}(n, p, q)$ be an Edge-Markovian Dynamic Network in its stationary distribution. If $p \geq c \frac{\log n}{n}$ for a sufficiently large $c$ then with high probability, the flooding time in $\mathcal{M}(n, p, q)$ is 

	$$
		\mathcal{O}\left(\frac{\log n}{\log np} + \log \log np \right)
	$$
\end{theorem}

TODO: Proof

\begin{proof}
	
\end{proof}

% TODO: What happens if p < c logn/n?
% lower bound on mixing time

\subsection{Simulations}

TODO: Evaluate tightness of bound for Edge-Markovian evolution

\section{Bounding the asynchronous rumour spreading time in terms of the Cheeger constant}
\label{AsyncCheegerBound}

\subsection{Model}

TODO: Define asynchronous model - explain impact of changes in comparison to the other model (in this model each edge has rate 1 instead of 1/deg(v), bound not dependent on degrees of nodes)


Each edge is associated with a Poisson Process of unit rate. When the Possion Processes fires, if one of the endpoints knows the rumour then both endpoints immediately become aware of the rumour. 
% TODO: CHANGE WORDING - what if nodes know rumour already

\begin{definition}
	Edge-boundary of a set of vertices $A \subseteq V$
	$$
		\partial A = \left\{ \{u, v\} \in E \mid u \in A, v \in V \setminus A \right\} 
	$$
\end{definition}

\begin{definition}
	Cheeger Constant of a graph $G$
	$$
		h(G) = \min_{A \subset V, 0 < |A| \leq \frac{n}{2}} \frac{|\partial A|}{|A|}
	$$

\end{definition}

\begin{theorem}
	Let $\mathcal{G} = (G_t)_{t \in \mathbb{N}}$ be an $n$-node asychronus PUSH-model dynamic network, where at least one node is aware of a rumour in $G_0$.

	Then, with high probability, the rumour will have spread to all the nodes of $\mathcal{G}$ in time at most

	$$
		\min \left\{t : \sum_{k=0}^t h(G_k) \geq C \log n \right\} 
	$$

	\noindent
	for a sufficiently large $C$
\end{theorem}

TODO: Explain how can apply the following result in the proof from the first asychronus model

\begin{proof}

	% TODO: Move tau' defn to lemma statement
	Let $\tau'$ be the first time at which the number of informed nodes increases by at least $\frac{m(\tau)}{2}$ 
	
	$$
		\tau' = \min\left\{\gamma : |I_{\gamma}| \geq |I_\tau| + \frac{m(\tau)}{2}\right\}
	$$
	Let $ \gamma \in [\tau, \tau')$.

	Suppose  $|I_\gamma| \leq \frac{n}{2}$. By the definition of the Cheeger constant we have that $h(G_\gamma) \leq \frac{|\partial I_\gamma|}{|I_\gamma|}$. Note that $ \partial I_\gamma = E(I_\gamma, U_\gamma)$, since $\partial I_\gamma$ is the set of edges with exactly one endpoint in $I_\gamma$, thus the other endpoint must be in the complement of $I_\gamma$, namely $U_\gamma$. Thus

	\begin{align*}
		\lambda(\gamma) &= |E(I_\gamma, U_\gamma)| \\
		& = |\partial I_\gamma| \\
		& \geq h(G_\gamma) |I_\gamma| \\
		& \geq h(G_\gamma) |I_\tau| & \text{since } |I_\gamma| \text{ is increasing} \\
		& \geq h(G_\gamma) \frac{m(\tau)}{2}
	\end{align*}
		
	%TODO: THIS JUST EXCLUDES THE ERROR CASE
	Now suppose $|I_\gamma| > \frac{n}{2}$, thus $|U_\gamma| < \frac{n}{2}$. Hence, by the definition of the Cheeger constant we have that $h(G_\gamma) \leq \frac{|\partial U_\gamma|}{|U_\gamma|}$. Note that $ \partial U_\gamma = E(I_\gamma, U_\gamma)$, since $\partial U_\gamma$ is the set of edges with exactly one endpoint in $U_\gamma$,  thus the other endpoint must be in the complement of $U_\gamma$, namely $I_\gamma$. Thus 

	\begin{align*}
		\lambda(\gamma) &= |E(I_\gamma, U_\gamma)| \\
		& = |\partial U_\gamma| \\
		& \geq h(G_\gamma) |U_\gamma| \\
	\end{align*} 

	Since $|I_t| + |U_t| = n$ for all $t$, we can reformulate the definition of $\tau'$ as follows

	$$
		\tau' = \min \left\{ \gamma : |U_\tau| - \frac{m(\tau)}{2} \geq |U_\gamma| \right\} 
	$$ 
	
	Since $\gamma < \tau'$ we have that

	\begin{align*}
		|U_\gamma| & \geq |U_\tau| - \frac{m(\tau)}{2} \\
		& \geq |U_\tau| - \frac{|U_\tau|}{2} \\
		& = \frac{|U_\tau|}{2} \\
	\end{align*}
	
	Hence $\lambda(\gamma) \geq h(G_\gamma)\frac{|U_\tau|}{2} = h(G_\gamma)\frac{m(\tau)}{2}$

\end{proof}

\subsection{Applying the bound to Edge-Markovian evolution}

TODO
\begin{enumerate}
	\item Prove that an individual G(n,p) ER graph satisfies $h(G) > p$ w.h.p
	\item Prove that the first log(n)/p ER graphs satisfy this property w.h.p (union bound on event at least one not satisfying)- see edge-markovian proof
	\item Combine with result from previous bound to get O(log(n)/p) concrete bound
	\item If possible - prove spread time is $\Theta(\log(n)/p)$
\end{enumerate}

\subsection{Simulations}

TODO: Evaluate bound with simulatons if can't prove spread time is  $\Theta(\log(n)/p)$

\section{Conclusion}

TODO:

\begin{enumerate}
	\item Summary of results
	\item Questions for further study
\end{enumerate}

\end{document}