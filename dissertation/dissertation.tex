\documentclass[a4paper,11pt]{article}

\usepackage{amsmath}
\usepackage{amssymb}
\usepackage{amsthm}
\usepackage{graphicx}
\usepackage{enumerate}
%you can add more packages using the same code above

%------------------

%\setlength{\topmargin}{0.0in}
%\setlength{\textheight}{10in}
%\setlength{\oddsidemargin}{0.0in}
%\setlength{\evensidemargin}{0.0in}
%\setlength{\textwidth}{6.5in}

%-------------------


\newtheorem{theorem}{Theorem}[section]
\newtheorem{proposition}[theorem]{Proposition}
\newtheorem{lemma}[theorem]{Lemma}
\newtheorem{corollary}[theorem]{Corollary}
\newtheorem{conjecture}[theorem]{Conjecture}


\theoremstyle{definition}
\newtheorem{definition}[theorem]{Definition}
\newtheorem*{example}{Example}

\newcommand*\comp[1]{\overline{#1}}

\newcommand{\lb}{\left}
\newcommand{\rb}{\right}

%------------------

%Everything before begin document is called the pre-amble and sets out how the document will look
%It is recommended you don't touch the pre-amble until you are familiar with LateX

\begin{document}

\begin{titlepage}
\thispagestyle{empty}

\begin{figure}[h]
\begin{center}
\includegraphics[scale=0.5]{graphics/uob_logo.pdf} %make sure the pdf file named 'uob' is saved in the same folder as this file
\end{center}
\end{figure}

\begin{center}
{\Large Bounding the Rumour-Spreading Time on Dynamic Networks\\ \vspace{1cm}Lucas O'Dowd-Jones}
\end{center}

\vspace{3cm}
\hrule
\begin{center}
Supervised by Ayalvadi Ganesh\\
Level M\\
40 Credit Points
\end{center}
\hrule

\vspace{3cm}
\begin{center}
\today
\end{center}

\end{titlepage}



\begin{abstract}
	Some sorts of documents need abstracts. Others do not.
\end{abstract}
	
	
\tableofcontents

\section{Definitions}

\begin{definition}
	Dynamic Network

	$$
		\mathcal{G} = (G_t)_{t \in \mathbb{N}}, G_t = (V, E_t)
	$$
\end{definition}

\begin{definition}
	Cut
	$$
		E(S, \comp{S}) = \left\{\{u, v\} \in E \mid u \in S, v \in \comp{S} \right\}
	$$
\end{definition}

\begin{definition}
	Diligence of a cut $ E(S, \comp{S}) $
	$$
		\rho(S) = \comp{d}(S) \min_{\{u, v\} \in E(S, \comp{S}) } \left\{ \max \left\{ \frac{1}{d_u},\frac{1}{d_v} \right\} \right\}
	$$ 

	where $\comp{d}(S) := \frac{\sum_{v \in S} d_v}{|S|}$ is the average degree of the vertices in $S$
\end{definition}

\begin{definition}
	Diligence of a graph $G$
	$$
		\rho(G) = \min_{S \in V} \rho(S) 
	$$
\end{definition}

\begin{definition}
	Absolute Diligence of a non-empty graph $G = (V, E)$
	$$
		\comp{\rho}(G) = \min_{\{u, v\}v \in E}\left\{ \max \left\{ \frac{1}{d_u},\frac{1}{d_v} \right\} \right\}
	$$
\end{definition}

\begin{definition}
	Conductance of a graph $G = (V, E)$

	$$
		\Phi(G) = \min_{\emptyset \neq S \subset V} \frac{|E(S, \comp{S})|}{\min\{\text{vol}(S), \text{vol}(\comp{S})\}}
	$$
\end{definition}

\begin{definition}
	Edge-Markovian Dynamic Network

	% TODO
\end{definition}

\section{Results}

\newcommand{\ModelIntro}{
	Let $\mathcal{G} = (G_t)_{t \in \mathbb{N}}$ be an $n$-node dynamic network, where at least one node is aware of a rumour in $G_0$.
}

\begin{theorem}
	\ModelIntro Then with high probability, the rumour will have spread to all the nodes of $\mathcal{G}$ in time at most

	$$
		\min \left\{t : \sum_{k=0}^t \Phi(G_k)\rho(G_k) \geq C \log n \right\} 
	$$
\end{theorem}

\begin{theorem}
	\ModelIntro Then with high probability, the rumour will have spread to all the nodes of $\mathcal{G}$ in time at most 

	$$ 
		\min \left\{ t : \sum_{k=0}^t \delta_{G_k} \comp{\rho}(G_k) \right\}
	$$
\end{theorem}

\begin{theorem}
	\ModelIntro Suppose also that there exists an increasing sequence $1 = h_0 \leq h_1 < \dots < h_s = \frac{n}{2}$ and decreasing sequence $k_1 \geq k_2 \geq \dots \geq k_s$ of positive real numbers, such that for all $t \in \mathbb{N}$, $G_t$ is a $(h_i, k_i)$-expander for every $i \in \{1, \dots , s\}$. Then the flooding time of $\mathcal{G}$ is

	$$
		\mathcal{O}\left(\sum_{i=1}^s \frac{\log \frac{h_i}{h_{i-1}}}{\log(1+k_i)}\right)
	$$
\end{theorem}

\begin{theorem}
	Let $\mathcal{M}(n, p, q)$ be an Edge-Markovian Dynamic Network in its stationary distribution. If $p \geq c \frac{\log n}{n}$ for a sufficiently large $c$ then with high probability, the flooding time in $\mathcal{M}(n, p, q)$ is 

	$$
		\mathcal{O}\left(\frac{\log n}{\log np} + \log \log np \right)
	$$
\end{theorem}

\section{Introduction}

Start your document with words, written in full sentences and paragraphs.
%Using the percentage symbol, you can include comments in your code that do not appear in the output.
It is a good idea to break your document into sections and subsections

\section{Formating}

We can \emph{emphasis} some words, i.e., make them \emph{italic}, and we can make some words \textbf{bold}.
Note how using a new line in the code does not correspond to a new line in the output file.
Same if we have        a           large                white                   space.

Instead, if we want a new line/new paragraph, you need to press enter twice, or use \\
which starts a new line but not a new paragraph.

\subsection{lists}

Lists can be numbered or ununmbered, and you can have sub-list inside a list.

\begin{enumerate}
	\item This is the first item in a numbered list.

	\item And the second
	
	\item 
	\begin{enumerate}
		\item Here the third item is in fact a numbered sub-list.
		\item item 2 of the numbered sub-list
	\end{enumerate}

	\item 
	\begin{itemize}
		\item Here the fourth item is an unnumbered sub-list.
		\item item 2 of the unnumbered sub-list
	\end{itemize}
\end{enumerate}

\subsection{Definitions and theorems}

Definitions, theorems, lemmas and so on, are 'enviroments' (like documents and lists). They need to begin and end.

\begin{definition}\label{my_def}
	A \emph{label} allows the user to tell Latex 'remember the numbering of that definition/theorem/equation'
\end{definition}

\begin{lemma} \label{my_lem}
	If something has a label, then we can refer to it, without knowing what number it is 
\end{lemma}

\begin{proof}
	For example, by calling up Definition . This works even if the ordering of things move.
	Note that the end of proof square box is already there
\end{proof}

\begin{theorem}
	And a final theorem
\end{theorem}

\begin{proof}
	Combining Definition  with Lemman we get Equation  below.
\end{proof}

\section{Including maths}

Some maths, like $\varepsilon>0$ or $a_{23}=\alpha^3$, is written in-line. More important or complex maths is displayed on its own line.
For example, $$ \lim_{x\to\infty}f(x)=\frac{\pi}{4}.$$

Sometimes you need multiple lines of maths to line up nicely:

\begin{align*}
f(x+y)&=(x+y,-2(x+y))\\
&=(x,-2x)+(y,-2y)\\
&=f(x)+f(y),
\end{align*}

and sometimes you want to number lines in an equation

\begin{align}
A^{T} & =\begin{pmatrix}1 & 2\\
3 & 4
\end{pmatrix}^{T}\\
\label{my_eqn}  & =\begin{pmatrix}1 & 3\\
2 & 4
\end{pmatrix}
\end{align}






\begin{thebibliography}{99}

	\bibitem{lamport94}
	  Leslie Lamport,
	  \textit{\LaTeX: a document preparation system},
	  Addison Wesley, Massachusetts,
	  2nd edition,
	  1994.
	  
	\bibitem{referencing}
		Wikibooks,
		\textit{LaTeX/Bibliography Management},
		[0nline],
		Accessed at https://en.wikibooks.org/wiki/LaTeX/Bibliography\_Management,
		(DATE ACCESSED).
		
	
	\end{thebibliography}

\end{document}