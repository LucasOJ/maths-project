\documentclass[a4paper,11pt]{article}

\usepackage{amsmath}
\usepackage{amssymb}
\usepackage{amsthm}
\usepackage{graphicx}
\usepackage{enumerate}
%you can add more packages using the same code above

%------------------

%\setlength{\topmargin}{0.0in}
%\setlength{\textheight}{10in}
%\setlength{\oddsidemargin}{0.0in}
%\setlength{\evensidemargin}{0.0in}
%\setlength{\textwidth}{6.5in}

%-------------------


\newtheorem{theorem}{Theorem}[section]
\newtheorem{proposition}[theorem]{Proposition}
\newtheorem{lemma}[theorem]{Lemma}
\newtheorem{corollary}[theorem]{Corollary}
\newtheorem{conjecture}[theorem]{Conjecture}


\theoremstyle{definition}
\newtheorem{definition}[theorem]{Definition}
\newtheorem*{example}{Example}

\newcommand*\comp[1]{\overline{#1}}

\newcommand{\lb}{\left}
\newcommand{\rb}{\right}

%------------------

%Everything before begin document is called the pre-amble and sets out how the document will look
%It is recommended you don't touch the pre-amble until you are familiar with LateX

\begin{document}

\begin{titlepage}
\thispagestyle{empty}

\begin{figure}[h]
\begin{center}
\includegraphics[scale=0.5]{graphics/uob_logo.pdf} %make sure the pdf file named 'uob' is saved in the same folder as this file
\end{center}
\end{figure}

\begin{center}
{\Large Bounding the Rumour-Spreading Time on Dynamic Networks\\ \vspace{1cm}Lucas O'Dowd-Jones}
\end{center}

\vspace{3cm}
\hrule
\begin{center}
Supervised by Ayalvadi Ganesh\\
Level M\\
40 Credit Points
\end{center}
\hrule

\vspace{3cm}
\begin{center}
\today
\end{center}

\end{titlepage}



\begin{abstract}
	Some sorts of documents need abstracts. Others do not.
\end{abstract}
	
	
\tableofcontents

\section{Definitions}

\begin{definition}
	Dynamic Network

	$$
		\mathcal{G} = (G_t)_{t \in \mathbb{N}}, G_t = (V, E_t)
	$$
\end{definition}

\begin{definition}
	Cut
	$$
		E(S, \comp{S}) = \left\{\{u, v\} \in E \mid u \in S, v \in \comp{S} \right\}
	$$
\end{definition}

\begin{definition}
	Diligence of a cut $ E(S, \comp{S}) $
	$$
		\rho(S) = \comp{d}(S) \min_{\{u, v\} \in E(S, \comp{S}) } \left\{ \max \left\{ \frac{1}{d_u},\frac{1}{d_v} \right\} \right\}
	$$ 

	where $\comp{d}(S) := \frac{\sum_{v \in S} d_v}{|S|}$ is the average degree of the vertices in $S$
\end{definition}

\begin{definition}
	Diligence of a graph $G$
	$$
		\rho(G) = \min_{S \in V} \rho(S) 
	$$
\end{definition}

\begin{definition}
	Absolute Diligence of a non-empty graph $G = (V, E)$
	$$
		\comp{\rho}(G) = \min_{\{u, v\}v \in E}\left\{ \max \left\{ \frac{1}{d_u},\frac{1}{d_v} \right\} \right\}
	$$
\end{definition}

\begin{definition}
	Conductance of a graph $G = (V, E)$

	$$
		\Phi(G) = \min_{\emptyset \neq S \subset V} \frac{|E(S, \comp{S})|}{\min\{\text{vol}(S), \text{vol}(\comp{S})\}}
	$$
\end{definition}

\begin{definition}
	Edge-Markovian Dynamic Network

	% TODO
\end{definition}

\section{Introduction}

% Motivation for studying

% Results proved

% My contributions

\section{Bounding the asynchronous rumour spreading time in terms of the conductance}

\subsection{Possion Processes}

Superposition

\subsection{Order statistics of exponential random variables}

\newcommand{\ModelIntro}{
	Let $\mathcal{G} = (G_t)_{t \in \mathbb{N}}$ be an $n$-node dynamic network, where at least one node is aware of a rumour in $G_0$.
}

\begin{theorem}
	\ModelIntro Then, with high probability, the rumour will have spread to all the nodes of $\mathcal{G}$ in time at most

	$$
		\min \left\{t : \sum_{k=0}^t \Phi(G_k)\rho(G_k) \geq C \log n \right\} 
	$$

	\noindent
	for a sufficiently large $C$
\end{theorem}

\begin{proof}

\end{proof}

\begin{theorem}
	\ModelIntro Then with high probability, the rumour will have spread to all the nodes of $\mathcal{G}$ in time at most 

	$$ 
		\min \left\{ t : \sum_{k=0}^t \delta_{G_k} \comp{\rho}(G_k) \right\}
	$$

	%TODO: FINISH THEORM STATEMENT EG
	% for a sufficiently large $C$
\end{theorem}

\subsection{Interpretation}

% Why diligence useful?
% When useful/better than other bounds?
% Difference between absolute and non-absolute

\subsection{Simulations}

% Look at symmetrical graphs (complete, star, ring, path) and work evaluate bounds - verify they hold

%Simulations for random evolutions eg ER 

\begin{proof}
	
\end{proof}

\section{Bounding the asynchronous rumour spreading time in terms of the Cheeger constant}

\begin{definition}
	Edge-boundary of a set of vertices $A \subseteq V$
	$$
		\partial A = \left\{ \{u, v\} \in E \mid u \in A, v \in V \setminus A \right\} 
	$$
\end{definition}

\begin{definition}
	Cheeger Constant of a graph $G$
	$$
		h(G) = \min_{A \subset V, 0 < |A| \leq \frac{n}{2}} \frac{|\partial A|}{|A|}
	$$

\end{definition}

\begin{theorem}
	Let $\mathcal{G} = (G_t)_{t \in \mathbb{N}}$ be an $n$-node asychronus PUSH-model dynamic network, where at least one node is aware of a rumour in $G_0$. Each edge is associated with a Poisson Process of unit rate. When the Possion Processes fires, if one of the endpoints knows the rumour then both endpoints immediately become aware of the rumour. % TODO: CHANGE WORDING - what if nodes know rumour already

	Then, with high probability, the rumour will have spread to all the nodes of $\mathcal{G}$ in time at most

	$$
		\min \left\{t : \sum_{k=0}^t h(G_k) \geq C \log n \right\} 
	$$

	\noindent
	for a sufficiently large $C$
\end{theorem}

\begin{proof}

	% TODO: Move tau' defn to lemma statement
	Let $\tau'$ be the first time at which the number of informed nodes increases by at least $m(\tau)$ ie the smallest $\gamma$ such that $|I_{\gamma}| \geq |I_\tau| + m(\tau)$
	
	Let $ \gamma \in [\tau, \tau')$.

	Suppose  $|I_\gamma| \leq \frac{n}{2}$. By the definition of the Cheeger constant we have that $h(G_\gamma) \leq \frac{|\partial I_\gamma|}{|I_\gamma|}$. Note that $ \partial I_\gamma = E(I_\gamma, U_\gamma)$, since $\partial I_\gamma$ is the set of edges with has exactly one endpoint in $I_\gamma$, thus the other endpoint must be in the complement of $I_\gamma$, namely $U_\gamma$. Thus

	\begin{align*}
		\lambda(\gamma) &= |E(I_\gamma, U_\gamma)| \\
		& = |\partial I_\gamma| \\
		& \geq h(G_\gamma) |I_\gamma| \\
		& \geq h(G_\gamma) |I_\tau| & \text{since } |I_\gamma| \text{ is increasing} \\
		& \geq h(G_\gamma) m(\tau)
	\end{align*}
		
	%TODO: THIS JUST EXCLUDES THE ERROR CASE
	Now suppose $|I_\gamma| > \frac{n}{2}$, thus $|U_\gamma| < \frac{n}{2}$. Hence, by the definition of the Cheeger constant we have that $h(G_\gamma) \leq \frac{|\partial U_\gamma|}{|U_\gamma|}$. Note that $ \partial U_\gamma = E(I_\gamma, U_\gamma)$, since $\partial U_\gamma$ is the set of edges with exactly one endpoint in $U_\gamma$,  thus the other endpoint must be in the complement of $U_\gamma$, namely $I_\gamma$. Thus 

	\begin{align*}
		\lambda(\gamma) &= |E(I_\gamma, U_\gamma)| \\
		& = |\partial U_\gamma| \\
		& \geq h(G_\gamma) |U_\gamma| \\
	\end{align*} 

	Since $|I_t| + |U_t| = n$ for all $t$, we can reformulate the definition of $\tau'$ as follows

	\begin{definition}
		$\tau'$ is the smallest $\gamma$ such that $|U_\tau| - m(\tau) \geq |U_\gamma|$
	\end{definition}
	
	Since $\gamma < \tau'$ we have that

	\begin{align*}
		|U_\gamma| & \geq |U_\tau| - m(\tau) \\
		& \geq |U_\tau| - \frac{|U_\tau|}{2} \\
		& = \frac{|U_\tau|}{2} \\
	\end{align*}
	
	Hence $\lambda(\gamma) \geq h(G_\gamma)\frac{|U_\tau|}{2} = h(G_\gamma)m(\tau)$

\end{proof}

\subsection{Applying the bound to Edge-Markovian evolution}

For each edge there is a stationary distribution

Talk about stationarity

Why only need to consider stationary distribution?

Prove stationarity distribution of Edge-Markovian model is ER graph

Prove lower bound on $h(G)$ for ER graph

\section{Bounding the synchronous flooding time}

\begin{theorem}
	\ModelIntro Suppose also that there exists an increasing sequence $1 = h_0 \leq h_1 < \dots < h_s = \frac{n}{2}$ and decreasing sequence $k_1 \geq k_2 \geq \dots \geq k_s$ of positive real numbers, such that for all $t \in \mathbb{N}$, $G_t$ is a $(h_i, k_i)$-expander for every $i \in \{1, \dots , s\}$. Then the flooding time of $\mathcal{G}$ is

	$$
		\mathcal{O}\left(\sum_{i=1}^s \frac{\log \frac{h_i}{h_{i-1}}}{\log(1+k_i)}\right)
	$$
\end{theorem}

\begin{proof}
	
\end{proof}

\subsection{Applying the bound to Edge-Markovian evolution}

\begin{theorem}
	Let $\mathcal{M}(n, p, q)$ be an Edge-Markovian Dynamic Network in its stationary distribution. If $p \geq c \frac{\log n}{n}$ for a sufficiently large $c$ then with high probability, the flooding time in $\mathcal{M}(n, p, q)$ is 

	$$
		\mathcal{O}\left(\frac{\log n}{\log np} + \log \log np \right)
	$$
\end{theorem}

\begin{proof}
	
\end{proof}

% TODO: What happens if p < c logn/n?
% lower bound on mixing time

\subsection{Simulations}

\end{document}